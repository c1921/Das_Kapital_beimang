\documentclass{ctexbook}
\usepackage[a5paper,scale=0.8]{geometry}
\usepackage{graphicx}
\usepackage{pdfpages}
\usepackage[hidelinks]{hyperref}
\usepackage[namelimits]{amsmath}
\usepackage{amssymb}
\usepackage{amsfonts}
\usepackage{mathrsfs}
\usepackage{multirow}
\usepackage{fixltx2e}

\ctexset{
    part={
        name={第,篇},
        format+=\zihao{-0}
    },
    chapter={
        name={第,章},
        format+=\zihao{1}
    },
    section={
        name={,.},
        format+=\zihao{3},
        number=\arabic{section},
        aftername=\hspace{0pt}
    },
    subsection={
        name={,.},
        format+=\zihao{-3}\centering,
        number=\Alph{subsection},
        aftername=\hspace{0pt}
    },
    subsubsection={
        name={(,)},
        format+=\zihao{4}\centering,
        number=\arabic{subsubsection},
        aftername=\hspace{0pt}
    },
    paragraph={
        name={(,)},
        format+=\zihao{-4}\centering,
        number=\alph{paragraph},
        aftername=\hspace{0pt}
    }
}


\title{资本论}
\author{卡尔·马克思}


%————————————————————————————————————
%++++++++++++++++++++++++++++++++++++
%++++++++++++++++++++++++++++++++++++
%++++++++++++++++++++++++++++++++++++
%++++++++++++++++++++++++++++++++++++
%++++++++++++++++++++++++++++++++++++
%++++++++++++++++++++++++++++++++++++
%++++++++++++++++++++++++++++++++++++
%++++++++++++++++++++++++++++++++++++
%++++++++++++++++++++++++++++++++++++
%++++++++++++++++++++++++++++++++++++
%++++++++++++++++++++++++++++++++++++
%++++++++++++++++++++++++++++++++++++
%————————————————————————————————————

% 备忘录:
    % 1.罗马数字不能直接输入,需在 \romannumeral 后输入对应阿拉伯数字,可选择一个罗马数字后右键“更改所有匹配项”,“←”键一位输入阿拉伯数字,完成批量更改。
        % 大写 \uppercase\expandafter{\romannumeral}
        % 小写 \romannumeral
        
    % 2.Ctrl + Shift + L 选择所有匹配项,可用来选择段首空格快速缩进、空行等。

    % 3.待办事项标注为:%!以待搜索、解决。



\begin{document}

%————————————
% 封面

\begin{figure}[ht]
    \begin{center}
        \includepdf[width=\paperwidth]{figures/资本论封面.png}
    \end{center}
\end{figure}

\thispagestyle{empty}
\clearpage

%————————————
%全世界无产者,联合起来!

\vspace*{\fill}
    \begin{center}
        \large{\textcolor{red}{全世界无产者,联合起来!}}
    \end{center}
\vspace*{\fill}

\thispagestyle{empty}

\newpage
\mbox{}
\thispagestyle{empty}
\clearpage
\newpage


%————————————


\vspace*{\fill}
    \begin{center}
        \zihao{0}{\textbf{第一卷\\资本的生产过程}}
    \end{center}
\vspace*{\fill}

\thispagestyle{empty}

\newpage
\mbox{}
\thispagestyle{empty}
\clearpage
\newpage




%————————————
% 目录

\setcounter{tocdepth}{5}% 设置目录级数
\setcounter{secnumdepth}{4}% 设置在几级目录前标记序号

\setcounter{page}{1}
\pagenumbering{Roman}

\tableofcontents

\clearpage

%————————————
%信

\vspace*{\fill}
    \begin{center}
        马克思1876年8月16日给恩格斯的信
    \end{center}
\vspace*{\fill}

\thispagestyle{empty}
\clearpage

%————
%信

\vspace*{\fill}
    \begin{center}
        \large{献\hspace{2em}给\\
        \texttt{我的不能忘记的朋友\\
        勇敢的忠实的高尚的无产阶级先锋战士}}\\
        \LARGE{威廉•沃尔弗}

        \normalsize{~\\ 

        1809年6月21日生于塔尔瑙\\
        1864年5月9日死于曼彻斯特流亡生活中}
    \end{center}
\vspace*{\fill}

\thispagestyle{empty}
\clearpage

%————————————
%序跋
\frontmatter

\chapter[卡尔·马克思\hspace{1em}第一版序言]{第一版序言\\{\small 卡尔·马克思}}

现在我把这部著作的第一卷交给读者。这部著作是我1859年发表的《政治经济学批判》的续篇。初篇和续篇相隔很久,是由于多年的疾病一再中断了我的工作。

前书的内容已经概述在这一卷的第一章中[2]。这样做不仅是为了联贯和完整,叙述方式也改进了。在情况许可的范围内,前书只是略略提到的许多论点,这里都作了进一步的阐述;相反地,前书已经详细阐述的论点,这里只略略提到。关于价值理论和货币理论的历史的部分,现在自然完全删去了。但是前书的读者可以在本书第一章的注释中,找到有关这两种理论的历史的新材料。

万事开头难,每门科学都是如此。所以本书第一章,特别是分析商品的部分,是最难理解的。其中对价值实体和价值量的分析,我已经尽可能地做到通俗易懂\footnote{这样做之所以更加必要,是因为甚至斐·拉萨尔著作中反对舒尔采-德里奇的部分,即他声称已经提出我对那些问题的阐述的“思想精髓”的部分[3],也包含着严重的误解。顺便说一下,斐·拉萨尔经济著作中所有一般的理论原理,如关于资本的历史性质、关于生产关系和生产方式之间的联系等等,几乎是逐字地——甚至包括我创造的术语——从我的作品中抄去的,而且没有说明出处,这样做显然是出于宣传上的考虑。我当然不是说他在细节上的论述和实际上的应用,这同我没有关系。}。以货币形式为其完成形态的价值形式,是极无内容和极其简单的。然而,两千多年来人类智慧在这方面进行探讨的努力,并未得到什么结果,而对更有内容和更复杂的形式的分析,却至少已接近于成功。为什么会这样呢?因为已经发育的身体比身体的细胞容易研究些。并且,分析经济形式,既不能用显微镜,也不能用化学试剂。二者都必须用抽象力来代替。而对资产阶级社会说来,劳动产品的商品形式,或者商品的价值形式,就是经济的细胞形式。在浅薄的人看来,分析这种形式好象是斤斤于一些琐事。这的确是琐事,但这是显微镜下的解剖所要做的那种琐事。

因此,除了价值形式那一部分外,不能说这本书难懂。当然,我指的是那些想学到一些新东西、因而愿意自己思考的读者。

物理学家是在自然过程表现得最确实、最少受干扰的地方考察自然过程的,或者,如有可能,是在保证过程以其纯粹形态进行的条件下从事实验的。我要在本书研究的,是资本主义生产方式以及和它相适应的生产关系和交换关系。到现在为止,这种生产方式的典型地点是英国。因此,我在理论阐述上主要用英国作为例证。但是,如果德国读者看到英国工农业工人所处的境况而伪善地耸耸肩膀,或者以德国的情况远不是那样坏而乐观地自我安慰,那我就要大声地对他说:这正是说的阁下的事情![4]

问题本身并不在于资本主义生产的自然规律所引起的社会对抗的发展程度的高低。问题在于这些规律本身,在于这些以铁的必然性发生作用并且正在实现的趋势。工业较发达的国家向工业较不发达的国家所显示的,只是后者未来的景象。

撇开这点不说。在资本主义生产已经在我们那里完全确立的地方,例如在真正的工厂里,由于没有起抗衡作用的工厂法,情况比英国要坏得多。在其他一切方面,我们也同西欧大陆所有其他国家一样,不仅苦于资本主义生产的发展,而且苦于资本主义生产的不发展。除了现代的灾难而外,压迫着我们的还有许多遗留下来的灾难,这些灾难的产生,是由于古老的陈旧的生产方式以及伴随着它们的过时的社会关系和政治关系还在苟延残喘。不仅活人使我们受苦,而且死人也使我们受苦。死人抓住活人!

德国和西欧大陆其他国家的社会统计,与英国相比是很贫乏的。然而它还是把帷幕稍稍揭开,使我们刚刚能够窥见幕内美杜莎的头。如果我国各邦政府和议会象英国那样,定期指派委员会去调查经济状况,如果这些委员会象英国那样,有全权去揭发真相,如果为此能够找到象英国工厂视察员、编写《公共卫生》报告的英国医生、调查女工童工受剥削的情况以及居住和营养条件等等的英国调查委员那样内行、公正、坚决的人们,那末,我国的情况就会使我们大吃一惊。柏修斯需要一顶隐身帽来追捕妖怪。我们却用隐身帽紧紧遮住眼睛和耳朵,以便有可能否认妖怪的存在。

决不要在这上面欺骗自己。正象十八世纪美国独立战争给欧洲中产阶级敲起了警钟一样,十九世纪美国南北战争又给欧洲工人阶级敲起了警钟。在英国,变革过程已经十分明显。它达到一定程度后,一定会波及大陆。在那里,它将采取较残酷的还是较人道的形式,那要看工人阶级自身的发展程度而定。所以,现在的统治阶级,不管有没有较高尚的动机,也不得不为了自己的切身利益,把一切可以由法律控制的、妨害工人阶级发展的障碍除去。因此,我在本卷中用了很大的篇幅来叙述英国工厂法的历史、内容和结果。一个国家应该而且可以向其他国家学习。一个社会即使探索到了本身运动的自然规律,——本书的最终目的就是揭示现代社会的经济运动规律,——它还是既不能跳过也不能用法令取消自然的发展阶段。但是它能缩短和减轻分娩的痛苦。

为了避免可能产生的误解,要说明一下。我决不用玫瑰色描绘资本家和地主的面貌。不过这里涉及到的人,只是经济范畴的人格化,是一定的阶级关系和利益的承担者。我的观点是:社会经济形态的发展是一种自然历史过程。不管个人在主观上怎样超脱各种关系,他在社会意义上总是这些关系的产物。同其他任何观点比起来,我的观点是更不能要个人对这些关系负责的。

在政治经济学领域内,自由的科学研究遇到的敌人,不只是它在一切其他领域内遇到的敌人。政治经济学所研究的材料的特殊性,把人们心中最激烈、最卑鄙、最恶劣的感情,把代表私人利益的复仇女神召唤到战场上来反对自由的科学研究。例如,英国高教会宁愿饶恕对它的三十九个信条中的三十八个信条展开的攻击,而不饶恕对它的现金收入的三十九分之一进行的攻击。在今天,同批评传统的财产关系相比,无神论本身是一种很轻的罪。但在这方面,进步仍然是无可怀疑的。以最近几星期内发表的蓝皮书[5]《关于工业和工联问题同女王陛下驻外公使馆的通讯》为例。英国女王驻外使节在那里坦率地说,在德国,在法国,一句话,在欧洲大陆的一切文明国家,现有的劳资关系的变革同英国一样明显,一样不可避免。同时,大西洋彼岸的美国副总统威德先生也在公众集会上说:在奴隶制废除后,资本关系和土地所有权关系的变革会提到日程上来!这是时代的标志,不是用紫衣黑袍遮掩得了的。这并不是说明天就会出现奇迹。但这表明,甚至在统治阶级中间也已经透露出一种模糊的感觉:现在的社会不是坚实的结晶体,而是一个能够变化并且经常处于变化过程中的机体。

这部著作的第二卷将探讨资本的流通过程(第二册)和总过程的各种形式(第三册),第三卷即最后一卷(第四册)将探讨理论史。

任何的科学批评的意见我都是欢迎的。而对于我从来就不让步的所谓舆论的偏见,我仍然遵守伟大的佛罗伦萨诗人的格言:

走你的路,让人们去说罢![6]

\begin{flushright}
    \textbf{卡尔·马克思}\\
    \small{1867年7月25日于伦敦}
\end{flushright}



\chapter[卡尔·马克思\hspace{1em}第二版跋]{第二版跋\\{\small 卡尔·马克思}}

我首先应当向第一版的读者指出第二版中所作的修改。很明显的是,篇目更加分明了。各处新加的注,都标明是第二版注。就正文说,最重要的有下列各点:

第一章第一节更加科学而严密地从表现每个交换价值的等式的分析中引出了价值,而且明确地突出了在第一版中只是略略提到的价值实体和由社会必要劳动时间决定的价值量之间的联系。第一章第三节(价值形式)全部改写了,第一版的双重叙述就要求这样做。——顺便指出,这种双重叙述是我的朋友,汉诺威的路·库格曼医生建议的。1867年春,初校样由汉堡寄来时,我正好访问他。他劝我说,大多数读者需要有一个关于价值形式的更带讲义性的补充说明。——第一章最后一节《商品的拜物教性质及其秘密》大部分修改了。第三章第一节(价值尺度)作了详细的修改,因为在第一版中,考虑到《政治经济学批判》(1859年柏林版)已有的说明,这一节是写得不够细致的。第七章,特别是这一章的第二节,作了很大的修改。

原文中局部的、往往只是修辞上的修改,用不着一一列举出来。这些修改全书各处都有。但是,现在我校阅要在巴黎出版的法译本时,发现德文原本某些部分需要更彻底地修改,某些部分需要更好地修辞或更仔细地消除一些偶然的疏忽。可是我没有时间这样做,因为只是在1871年秋,正当我忙于其他迫切的工作的时候,我才接到通知说,书已经卖完了,而第二版在1872年1月就要付印。

《资本论》在德国工人阶级广大范围内迅速得到理解,是对我的劳动的最好的报酬。一个在经济方面站在资产阶级立场上的人,维也纳的工厂主迈尔先生,在普法战争期间发行的一本小册子中[7]说得很对:被认为是德国世袭财产的卓越的理论思维能力,已在德国的所谓有教养的阶级中完全消失了,但在德国工人阶级中复活了。[8]

在德国,直到现在,政治经济学一直是外来的科学。古斯达夫·冯·居利希在他的《商业、工业和农业的历史叙述》中,特别是在1830年出版的该书的前两卷中,已经大体上谈到了妨碍我国资本主义生产方式发展、因而也妨碍我国现代资产阶级社会建立的历史条件。可见,政治经济学在我国缺乏生存的基础。它作为成品从英国和法国输入;德国的政治经济学教授一直是学生。别国的现实在理论上的表现,在他们手中变成了教条集成,被他们用包围着他们的小资产阶级世界的精神去解释,就是说,被曲解了。他们不能把在科学上无能为力的感觉完全压制下去,他们不安地意识到,他们必须在一个实际上不熟悉的领域内充当先生,于是就企图用博通文史的美装,或用无关材料的混合物来加以掩饰。这种材料是从所谓官房学——各种知识的杂拌,满怀希望的\footnote{第3版和第4版中是:毫无希望的。——编者注}德国官僚候补者必须通过的炼狱之火——抄袭来的。

从1848年起,资本主义生产在德国迅速地发展起来,现在正是它的欺诈盛行的时期。但是我们的专家还是命运不好。当他们能够公正无私地研究政治经济学时,在德国的现实中没有现代的经济关系。而当这种关系出现时,他们所处的境况已经不再容许他们在资产阶级的视野之内进行公正无私的研究了。只要政治经济学是资产阶级的政治经济学,就是说,只要它把资本主义制度不是看作历史上过渡的发展阶段,而是看作社会生产的绝对的最后的形式,那就只有在阶级斗争处于潜伏状态或只是在个别的现象上表现出来的时候,它还能够是科学。

拿英国来说。英国古典政治经济学是属于阶级斗争不发展的时期的。它的最后的伟大的代表李嘉图,终于有意识地把阶级利益的对立、工资和利润的对立、利润和地租的对立当作他的研究的出发点,因为他天真地把这种对立看作社会的自然规律。这样,资产阶级的经济科学也就达到了它的不可逾越的界限。还在李嘉图活着的时候,就有一个和他对立的人西斯蒙第批判资产阶级的经济科学了。\footnote{见我的《政治经济学批判》第39页[9]。}

随后一个时期,从1820年到1830年,在英国,政治经济学方面的科学活动极为活跃。这是李嘉图的理论庸俗化和传播的时期,同时也是他的理论同旧的学派进行斗争的时期。这是一场出色的比赛。当时的情况,欧洲大陆知道得很少,因为论战大部分是分散在杂志论文、关于时事问题的著作和抨击性小册子上。这一论战的公正无私的性质——虽然李嘉图的理论也例外地被用作攻击资产阶级经济的武器——可由当时的情况来说明。一方面,大工业刚刚脱离幼年时期;大工业只是从1825年的危机才开始它的现代生活的周期循环,就证明了这一点。另一方面,资本和劳动之间的阶级斗争被推到后面:在政治方面是由于纠合在神圣同盟周围的政府和封建主同资产阶级所领导的人民大众之间发生了纠纷;在经济方面是由于工业资本和贵族土地所有权之间发生了纷争。这种纷争在法国是隐藏在小块土地所有制和大土地所有制的对立后面,在英国则在谷物法颁布后公开爆发出来。这个时期英国的政治经济学文献,使人想起魁奈医生逝世后法国经济学的狂飙时期,但这只是象晚秋晴日使人想起春天一样。1830年,最终决定一切的危机发生了。

法国和英国的资产阶级夺得了政权。从那时起,阶级斗争在实践方面和理论方面采取了日益鲜明的和带有威胁性的形式。它敲响了科学的资产阶级经济学的丧钟。现在问题不再是这个或那个原理是否正确,而是它对资本有利还是有害,方便还是不方便,违背警章还是不违背警章。不偏不倚的研究让位于豢养的文丐的争斗,公正无私的科学探讨让位于辩护士的坏心恶意。甚至以工厂主科布顿和布莱特为首的反谷物法同盟[10]抛出的强迫人接受的小册子,由于对地主贵族展开了论战,即使没有科学的意义,毕竟也有历史的意义。但是从罗伯特·皮尔爵士执政以来,这最后一根刺也被自由贸易的立法从庸俗经济学那里拔掉了。

1848年大陆的革命也在英国产生了反应。那些还要求有科学地位、不愿单纯充当统治阶级的诡辩家和献媚者的人,力图使资本的政治经济学同这时已不容忽视的无产阶级的要求调和起来。于是,以约翰·斯图亚特·穆勒为最著名代表的毫无生气的混合主义产生了。这宣告了“资产阶级”经济学的破产,关于这一点,俄国的伟大学者和批评家尼·车尔尼雪夫斯基在他的《穆勒政治经济学概述》中已作了出色的说明。

可见,在资本主义生产方式的对抗性质在法英两国通过历史斗争而明显地暴露出来以后,资本主义生产方式才在德国成熟起来,同时,德国无产阶级比德国资产阶级在理论上已经有了更明确的阶级意识。因此,当资产阶级政治经济学作为一门科学看来在德国有可能产生的时候,它又成为不可能了。

在这种情况下,资产阶级政治经济学的代表人物分成了两派。一派是精明的、贪利的实践家,他们聚集在庸俗经济学辩护论的最浅薄的因而也是最成功的代表巴师夏的旗帜下。另一派是以经济学教授资望自负的人,他们追随约·斯·穆勒,企图调和不能调和的东西。德国人在资产阶级经济学衰落时期,也同在它的古典时期一样,始终只是学生、盲从者和模仿者,是外国大商行的小贩。

所以,德国社会特殊的历史发展,排除了“资产阶级”经济学在德国取得任何独创的成就的可能性,但是没有排除对它进行批判的可能性。就这种批判代表一个阶级而论,它能代表的只是这样一个阶级,这个阶级的历史使命是推翻资本主义生产方式和最后消灭阶级。这个阶级就是无产阶级。

德国资产阶级的博学的和不学无术的代言人,最初企图象他们在对付我以前的著作时曾经得逞那样,用沉默置《资本论》于死地。当这种策略已经不再适合时势的时候,他们就借口批评我的书,开了一些单方来“镇静资产阶级的意识”,但是他们在工人报刊上(例如约瑟夫·狄慈根在《人民国家报》上发表的文章[11])遇到了强有力的对手,至今还没有对这些对手作出答复。\footnote{德国庸俗经济学的油嘴滑舌的空谈家,指责我的著作的文体和叙述方法。没有人会比我本人更严厉地评论《资本论》的文字上的缺点。然而,为了使这些先生及其读者受益和愉快,我要在这里援引一篇英国的和一篇俄国的评论。同我的观点完全敌对的《星期六评论》在其关于德文第一版的短评中说道:叙述方法“使最枯燥无味的经济问题具有一种独特的魅力”。1872年4月20日的《圣彼得堡消息报》也说:“除了少数太专门的部分以外,叙述的特点是通俗易懂,明确,尽管研究对象的科学水平很高却非常生动。在这方面,作者……和大多数德国学者大不相同,这些学者……用含糊不清、枯燥无味的语言写书,以致普通人看了脑袋都要裂开。”但是,对现代德国民族主义自由主义教授的著作的读者说来,要裂开的是和脑袋完全不同的东西。}

1872年春,彼得堡出版了《资本论》的优秀的俄译本。初版三千册现在几乎已售卖一空。1871年,基辅大学政治经济学教授尼·季别尔先生在他的《李嘉图的价值和资本的理论》一书中就已经证明,我的价值、货币和资本的理论就其要点来说是斯密—李嘉图学说的必然的发展。使西欧读者在阅读他的这本出色的著作时感到惊异的,是纯理论观点的始终一贯。
人们对《资本论》中应用的方法理解得很差,这已经由各种互相矛盾的评论所证明。

例如,巴黎的《实证论者评论》[12]一方面责备我形而上学地研究经济学,另一方面责备我——你们猜猜看!——只限于批判地分析既成的事实,而没有为未来的食堂开出调味单(孔德主义的吗?)。关于形而上学的责备,季别尔教授指出:

\texttt{“就理论本身来说,马克思的方法是整个英国学派的演绎法,其优点和缺点是一切最优秀的理论经济学家所共有的。”}

莫·布洛克先生在《德国的社会主义理论家》(摘自1872年7月和8月《经济学家杂志》)一文中,指出我的方法是分析的方法,他说:

\texttt{“马克思先生通过这部著作而成为一个最出色的具有分析能力的思想家”。}

德国的评论家当然大叫什么黑格尔的诡辩。彼得堡的《欧洲通报》在专谈《资本论》的方法一文(1872年5月号第427—436页[14])中,认为我的研究方法是严格的现实主义的,而叙述方法不幸是德国辩证法的。作者写道:

\texttt{“如果从外表的叙述形式来判断,那末最初看来,马克思是最大的唯心主义哲学家,而且是德国的即坏的唯心主义哲学家。而实际上,在经济学的批判方面,他是他的所有前辈都无法比拟的现实主义者……决不能把他称为唯心主义者。”}

我回答这位作者先生的最好的办法,是从他自己的批评中摘出几段话来,这几段话也会使某些不懂俄文原文的读者感到兴趣。

这位作者先生从我的《政治经济学批判》序言(1859年柏林版第4—7页[15],在那里我说明了我的方法的唯物主义基础)中摘引一段话后说:

\texttt{“在马克思看来,只有一件事情是重要的,那就是发现他所研究的那些现象的规律。而且他认为重要的,不仅是在这些现象具有完成形式和处于一定时期内可见到的联系中的时候支配着它们的那种规律。在他看来,除此而外,最重要的是这些现象变化的规律,这些现象发展的规律,即它们由一种形式过渡到另一种形式,由一种联系秩序过渡到另一种联系秩序的规律。他一发现了这个规律,就详细地来考察这个规律在社会生活中表现出来的各种后果……所以马克思竭力去做的只是一件事:通过准确的科学研究来证明一定的社会关系秩序的必然性,同时尽可能完善地指出那些作为他的出发点和根据的事实。为了这个目的,只要证明现有秩序的必然性,同时证明这种秩序不可避免地要过渡到另一种秩序的必然性就完全够了,而不管人们相信或不相信,意识到或没有意识到这种过渡。马克思把社会运动看作受一定规律支配的自然历史过程,这些规律不仅不以人的意志、意识和意图为转移,反而决定人的意志、意识和意图……既然意识要素在文化史上只起着这种从属作用,那末不言而喻,以文化本身为对象的批判,比任何事情更不能以意识的某种形式或某种结果为依据。这就是说,作为这种批判的出发点的不能是观念,而只能是外部的现象。批判将不是把事实和观念比较对照,而是把一种事实同另一种事实比较对照。对这种批判唯一重要的是,把两种事实尽量准确地研究清楚,使之真正形成相互不同的发展阶段,但尤其重要的是,同样准确地把各种秩序的序列、把这些发展阶段所表现出来的联贯性和联系研究清楚……但是有人会说,经济生活的一般规律,不管是应用于现在或过去,都是一样的。马克思否认的正是这一点。在他看来,这样的抽象规律是不存在的……根据他的意见,恰恰相反,每个历史时期都有它自己的规律。一旦生活经过了一定的发展时期,由一定阶段进入另一阶段时,它就开始受另外的规律支配。总之,经济生活呈现出的现象,和生物学的其他领域的发展史颇相类似……旧经济学家不懂得经济规律的性质,他们把经济规律同物理学定律和化学定律相比拟……对现象所作的更深刻的分析证明,各种社会机体象动植物机体一样,彼此根本不同……由于各种机体的整个结构不同,它们的各个器官有差别,以及器官借以发生作用的条件不一样等等,同一个现象却受完全不同的规律支配。例如,马克思否认人口规律在任何时候在任何地方都是一样的。相反地,他断言每个发展阶段有它自己的人口规律……生产力的发展水平不同,生产关系和支配生产关系的规律也就不同。马克思给自己提出的目的是,从这个观点出发去研究和说明资本主义经济制度,这样,他只不过是极其科学地表述了任何对经济生活进行准确的研究必须具有的目的……这种研究的科学价值在于阐明了支配着一定社会机体的产生、生存、发展和死亡以及为另一更高的机体所代替的特殊规律。马克思的这本书确实具有这种价值”。}

这位作者先生把他称为我的实际方法的东西描述得这样恰当,并且在考察我个人对这种方法的运用时又抱着这样的好感,那他所描述的不正是辩证方法吗?

当然,在形式上,叙述方法必须与研究方法不同。研究必须充分地占有材料,分析它的各种发展形式,探寻这些形式的内在联系。只有这项工作完成以后,现实的运动才能适当地叙述出来。这点一旦做到,材料的生命一旦观念地反映出来,呈现在我们面前的就好象是一个先验的结构了。
我的辩证方法,从根本上来说,不仅和黑格尔的辩证方法不同,而且和它截然相反。在黑格尔看来,思维过程,即他称为观念而甚至把它变成独立主体的思维过程,是现实事物的创造主,而现实事物只是思维过程的外部表现。我的看法则相反,观念的东西不外是移入人的头脑并在人的头脑中改造过的物质的东西而已。

将近三十年以前,当黑格尔辩证法还很流行的时候,我就批判过黑格尔辩证法的神秘方面。但是,正当我写《资本论》第一卷时,愤懑的、自负的、平庸的、今天在德国知识界发号施令的模仿者们[16],却已高兴地象莱辛时代大胆的莫泽斯·门德尔森对待斯宾诺莎那样对待黑格尔,即把他当作一条“死狗”了。因此,我要公开承认我是这位大思想家的学生,并且在关于价值理论的一章中,有些地方我甚至卖弄起黑格尔特有的表达方式。辩证法在黑格尔手中神秘化了,但这决不妨碍他第一个全面地有意识地叙述了辩证法的一般运动形式。在他那里,辩证法是倒立着的。必须把它倒过来,以便发现神秘外壳中的合理内核。

辩证法,在其神秘形式上,成了德国的时髦东西,因为它似乎使现存事物显得光彩。辩证法,在其合理形态上,引起资产阶级及其夸夸其谈的代言人的恼怒和恐怖,因为辩证法在对现存事物的肯定的理解中同时包含对现存事物的否定的理解,即对现存事物的必然灭亡的理解;辩证法对每一种既成的形式都是从不断的运动中,因而也是从它的暂时性方面去理解;辩证法不崇拜任何东西,按其本质来说,它是批判的和革命的。

使实际的资产者最深切地感到资本主义社会充满矛盾的运动的,是现代工业所经历的周期循环的变动,而这种变动的顶点就是普遍危机。这个危机又要临头了,虽然它还处于预备阶段;由于它的舞台的广阔和它的作用的强烈,它甚至会把辩证法灌进新的神圣普鲁士德意志帝国的暴发户们的头脑里去。

\begin{flushright}
    \textbf{卡尔·马克思}\\
    \small{1873年1月24日于伦敦}
\end{flushright}






\chapter[卡尔·马克思\hspace{1em}法文版序言]{法文版序言\\{\small 卡尔·马克思}}

致莫里斯·拉沙特尔公民

亲爱的公民:

您想定期分册出版《资本论》的译本,我很赞同。这本书这样出版,更容易到达工人阶级的手里,在我看来,这种考虑是最为重要的。

这是您的想法好的一面,但也有坏的一面:我所使用的分析方法至今还没有人在经济问题上运用过,这就使前几章读起来相当困难。法国人总是急于追求结论,渴望知道一般原则同他们直接关心的问题的联系,因此我很担心,他们会因为一开始就不能继续读下去而气馁。

这是一种不利,对此我没有别的办法,只有事先向追求真理的读者指出这一点,并提醒他们。在科学上没有平坦的大道,只有不畏劳苦沿着陡峭山路攀登的人,才有希望达到光辉的顶点。
亲爱的公民,请接受我对您的忠诚。

\begin{flushright}
    \textbf{卡尔·马克思}\\
    \small{1872年3月18日于伦敦}
\end{flushright}


\chapter[卡尔·马克思\hspace{1em}法文版跋]{法文版跋\\{\small 卡尔·马克思}}

约·鲁瓦先生保证尽可能准确地、甚至逐字逐句地进行翻译。他非常认真地完成了自己的任务。但正因为他那样认真,我不得不对表述方法作些修改,使读者更容易理解。由于本书分册出版,这些修改是逐日作的,所以不能处处一样仔细,文体不免有不一致的地方。

在担负校正工作后,我就感到作为依据的原本(德文第二版)应当作一些修改,有些论述要简化,另一些要加以完善,一些补充的历史材料或统计材料要加进去,一些批判性评注要增加,等等。不管这个法文版本有怎样的文字上的缺点,它仍然在原本之外有独立的科学价值,甚至对懂德语的读者也有参考价值。

下面是我从德文第二版跋中摘引的几段,是有关政治经济学在德国的发展和本书运用的方法的。

\begin{flushright}
    \textbf{卡尔·马克思}\\
    \small{1875年4月28日于伦敦}
\end{flushright}


\chapter[弗里德里希·恩格斯\hspace{1em}第三版序言]{第三版序言\\{\small 弗里德里希·恩格斯}}

马克思不幸已不能亲自进行这个第三版的付印准备工作。这位大思想家——现在,连反对他的人也拜服他的伟大了——已于1883年3月14日逝世。

我失去了一个相交四十年的最好的、最亲密的朋友,他给我的教益是无法用言语形容的。现在,不论出版这个第三版的任务,还是出版以手稿形式遗留下来的第二卷的任务,都落在我的身上了。在这里,我应该告诉读者,我是怎样履行前一项任务的。

马克思原想把第一卷原文大部分改写一下,把某些论点表达得更明确一些,把新的论点增添进去,把直到最近时期的历史材料和统计材料补充进去。由于他的病情和急于完成第二卷的定稿,他放弃了这一想法。他只作了一些最必要的修改,只把当时出版的法文版[17]中已有的增补收了进去。

在马克思的遗物中,我发现了一个德文本,其中有些地方他作了修改,标明何处应参看法文版;同时还发现了一个法文本,其中准确地标出了所要采用的地方。这些修改和增补,除少数外,都属于本书的最后一部分,即资本的积累过程那一篇。旧版的这一篇原文比其他各篇更接近于初稿,而前面各篇都作过比较彻底的修改。因此,这一篇的文体更加生动活泼,更加一气呵成,但也更不讲究,夹杂英文语气,有不明确的地方;叙述过程中间或有不足之处,因为个别重要论点只是提了一下。

说到文体,马克思亲自彻底校订了许多章节,并且多次作过口头指示,这就给了我一个标准去取舍英文术语和英文语气。马克思一定还会修改那些增补的地方,并且用他那精练的德语代替流畅的法语;而我只要把它们移译过来,尽量和原文协调一致,也就满足了。

因此,在这第三版中,凡是我不能确定作者自己是否会修改的地方,我一个字也没有改。我也没有想到把德国经济学家惯用的一些行话弄到《资本论》里面来。例如,这样一种费解的行话:把通过支付现金而让别人为自己劳动的人叫做劳动给予者,把为了工资而让别人取走自己的劳动的人叫做劳动受取者。法文travail〔劳动〕在日常生活中也有“职业”的意思。但是,如果有个经济学家把资本家叫做donneur de travail〔劳动给予者〕,把工人叫做receveur de travail〔劳动受取者〕,法国人当然会把他看作疯子。

我也不能把原文中到处使用的英制货币和度量衡单位换算成新德制单位。在第一版出版时,德制度量衡种类之多,犹如一年的天数那样,马克有两种(帝国马克当时还只存在于泽特贝尔的头脑中,这是他在三十年代末发明的),古尔登有两种,塔勒至少有三种,其中一种以“新三分之二”[18]为单位。在自然科学上通用的是公制度量衡,在世界市场上通用的是英制度量衡。在这种情况下,对于一部几乎完全要从英国的工业状况中取得实际例证的著作来说,采用英制计量单位是很自然的。这后一种理由直到今天还有决定意义,尤其因为世界市场上的有关情况几乎没有什么变化,而且正是在那些有决定意义的工业部门——制铁业和棉纺织业,至今通用的还几乎完全是英制度量衡。

最后,我说几句关于马克思的不大为人们了解的引证方法。在单纯叙述和描写事实的地方,引文(例如引用英国蓝皮书)自然是作为简单的例证。而在引证其他经济学家的理论观点的地方,情况就不同了。这种引证只是为了确定:一种在发展过程中产生的经济思想,是什么地方、什么时候、什么人第一次明确地提出的。这里考虑的只是,所提到的经济见解在科学史上是有意义的,能够多少恰当地从理论上表现当时的经济状况。至于这种见解从作者的观点来看是否还有绝对的或相对的意义,或者完全成为历史上的东西,那是毫无关系的。因此,这些引证只是从经济科学的历史中摘引下来作为正文的注解,从时间和首倡者两方面说明经济理论中各个比较重要的成就。这种工作在这样一种科学上是很必要的,这种科学的历史著作家们一直只是以怀有偏见、不学无术、追名逐利而著称。——现在我们也会明白,和第二版跋中所说的情况一样,为什么马克思只是在极例外的场合才引证德国经济学家的言论。

第二卷可望在1884年出版。

\begin{flushright}
    \textbf{弗里德思希·恩格斯}\\
    \small{1883年11月7日于伦敦}
\end{flushright}

\chapter[弗里德里希·恩格斯\hspace{1em}英文版序言]{英文版序言\\{\small 弗里德里希·恩格斯}}

关于《资本论》英译本的出版,不需要作任何解释了。但是鉴于本书阐述的理论几年前就已经为英美两国的定期刊物和现代著作经常提到,被攻击或辩护,被解释或歪曲,倒是需要说明一下为什么这个英译本延迟到今天才出版。

作者于1883年逝世后不久,我们就明显地感到这部著作确实需要一个英文版本,当时赛米尔·穆尔先生(马克思和本文作者多年的朋友,他可能比任何人都更熟悉这部著作)同意担任马克思的遗著处理人迫切希望出版的英译本的翻译工作。我们商定,由我对照原文校订译稿,并且在我认为适当的地方提出修改意见。但是后来,我们看到,穆尔先生本身的业务使他不能如我们大家所期待的那样很快完成翻译工作,于是我们欣然接受了艾威林博士的建议,由他担任一部分翻译工作。同时,马克思的小女儿艾威林夫人建议,由她核对引文,把引自英国作者和蓝皮书并由马克思译成德文的许多文句恢复成原文。除了少数无法避免的例外,她全部完成了这项工作。

本书下述各部分是艾威林博士翻译的:1.第十章(工作日)和第十一章(剩余价值率和剩余价值量);2.第六篇(工资,包括第十九章至第二十二章);3.第二十四章第四节(决定积累量的情况)至本书结尾,包括第二十四章最后一部分,第二十五章和第八篇全部(第二十六章至第三十三章);4.作者的两篇序言[19]。其余部分全是穆尔先生翻译的。因此,译者只对各自的译文负责,而我对整个工作负全部责任。

我们全部译文所依据的德文第三版,是我在1883年利用作者遗留的笔记整理的,笔记注明第二版的哪些地方应当改成1873年法文版标出的文句\footnote{《资本论》,卡尔·马克思著,莫·约·鲁瓦译,全文由作者校阅,由拉沙特尔在巴黎出版。这个译本,特别是该书的最后一部分,对德文第2版作了相当多的修改和补充。}。第二版原文中这样修改的地方,和马克思曾经为一个英译本(大约十年前在美国有人打算出版的一个英译本,但主要由于没有十分合适的译者而作罢)所写的许多书面指示中提出需要修改的地方大体相同。这份手稿是由我们的老朋友,新泽西州霍布根的弗·阿·左尔格提供给我们的。手稿指出,还有一些地方应该按照法文版进行补充;但是因为这份手稿是早在马克思对第三版作最后指示的前几年写的,所以我不敢随便利用它,除非在个别情况下,并且主要是在它有助于我们解决某些疑难问题的情况下才加以利用。而大多数有疑难问题的句子,我们也参考了法文本,因为它指出了,原文中某些有意义而在翻译中不得不舍弃的地方,作者自己也是打算舍弃的。

可是,有一个困难是我们无法为读者解除的。这就是:某些术语的应用,不仅同它们在日常生活中的含义不同,而且和它们在普通政治经济学中的含义也不同。但这是不可避免的。一门科学提出的每一种新见解,都包含着这门科学的术语的革命。化学是最好的例证,它的全部术语大约每二十年就彻底变换一次,几乎很难找到一种有机化合物不是先后拥有一系列不同的名称的。政治经济学通常满足于照搬工商业生活上的术语并运用这些术语,完全看不到这样做会使自己局限于这些术语所表达的观念的狭小范围。例如,古典政治经济学虽然完全知道,利润和地租都不过是工人必须向自己雇主提供的产品中无酬部分(雇主是这部分产品的第一个占有者,但不是它的最后的唯一的所有者)的一部分、一份,但即使这样,它也从来没有超出通常关于利润和地租的概念,从来没有把产品中这个无酬部分(马克思称它为剩余产品),就其总和即当作一个整体来研究过,因此,也从来没有对它的起源和性质,对制约着它的价值的以后分配的那些规律有一个清楚的理解。同样,一切产业,除了农业和手工业以外,都一概被包括在制造业(manufacture)这个术语中,这样,经济史上两个重大的本质不同的时期即以手工分工为基础的真正工场手工业时期和以使用机器为基础的现代工业时期的区别,就被抹杀了。不言而喻,把现代资本主义生产只看作是人类经济史上一个暂时阶段的理论所使用的术语,和把这种生产形式看作是永恒的最终阶段的那些作者所惯用的术语,必然是不同的。

关于作者的引证方法,不妨说几句。在大多数场合,也和往常一样,引文是用作证实文中论断的确凿证据。但在不少场合,引证经济学著作家的文句是为了证明:什么时候、什么地方、什么人第一次明确地提出某一观点。只要引用的论点具有重要意义,能够多少恰当地表现某一时期占统治地位的社会生产和交换条件,马克思就加以引证,至于马克思是否承认这种论点,或者说,这种论点是否具有普遍意义,那是完全没有关系的。因此,这些引证是从科学史上摘引下来并作为注解以充实正文的。

我们这个译本只包括这部著作的第一卷。但这第一卷是一部相当完整的著作,并且二十年来一直被当作一部独立的著作。1885年我用德文出版的第二卷,由于没有第三卷,显然是不完全的,而第三卷在1887年年底以前不能出版。到第三卷德文原稿刊行时,再考虑准备第二、三两卷的英文版也为时不晚。

《资本论》在大陆上常常被称为“工人阶级的圣经”。任何一个熟悉工人运动的人都不会否认:本书所作的结论日益成为伟大的工人阶级运动的基本原则,不仅在德国和瑞士是这样,而且在法国,在荷兰和比利时,在美国,甚至在意大利和西班牙也是这样;各地的工人阶级都越来越把这些结论看成是对自己的状况和自己的期望所作的最真切的表述。而在英国,马克思的理论正是在目前对社会主义运动产生着巨大的影响,这个运动在“有教养者”队伍中的传播,不亚于在工人阶级队伍中的传播。但这并不是一切。彻底研究英国的经济状况成为国民的迫切需要的时刻,很快就会到来。这个国家的工业体系的运动,——没有生产的从而没有市场的经常而迅速的扩大,这种运动就不可能进行,——已趋于停滞。自由贸易已经无计可施了;甚至曼彻斯特对自己这个昔日的经济福音也发生了怀疑[20]。迅速发展的外国工业,到处直接威胁着英国的生产,不仅在受关税保护的市场上,而且在中立市场上,甚至在英吉利海峡的此岸都是这样。生产力按几何级数增长,而市场最多也只是按算术级数扩大。1825年至1867年每十年反复一次的停滞、繁荣、生产过剩和危机的周期,看来确实已经结束,但这只是使我们陷入无止境的经常萧条的绝望泥潭。人们憧憬的繁荣时期将不再来临;每当我们似乎看到繁荣时期行将到来的种种预兆,这些预兆又消失了。而每一个冬天的来临都重新提出这一重大问题:“怎样对待失业者”;虽然失业人数年复一年地增加,却没有人解答这个问题;失业者再也忍受不下去,而要起来掌握自己命运的时刻,几乎指日可待了。毫无疑问,在这样的时刻,应当倾听这样一个人的声音,这个人的全部理论是他毕生研究英国的经济史和经济状况的结果,他从这种研究中得出这样的结论:至少在欧洲,英国是唯一可以完全通过和平的和合法的手段来实现不可避免的社会革命的国家。当然,他从来没有忘记附上一句话:他并不指望英国的统治阶级会不经过“维护奴隶制的叛乱”[21]而屈服在这种和平的和合法的革命面前。

\begin{flushright}
    \textbf{弗里德思希·恩格斯}\\
    \small{1886年11月5日}
\end{flushright}


\chapter[弗里德里希·恩格斯\hspace{1em}第四版序言]{第四版序言\\{\small 弗里德里希·恩格斯}}

第四版要求我尽可能把正文和注解最后确定下来。我是怎样实现这一要求的,可以简单说明如下:

根据再一次对照法文版和根据马克思亲手写的笔记,我又把法文版的一些地方补充到德文原文中去。这些补充是在第80页(第3版第88页)、第458—460页(第3版第509—510页)、第547—551页(第3版第600页)、第591—593页(第3版第644页)和第596页(第3版第648页)注79\footnote{见本卷第136、540-542、640-644、687-689、692-693页。——编者注。}。此外,我还按照法文版和英文版把一个很长的关于矿工的注解(第3版第509—515页)移入正文(第4版第461—467页)\footnote{见本卷第542-549页。——编者注}。其他一些小改动都是纯技术性的。

其次,我还补加了一些说明性的注释,特别是在那些由于历史情况的改变看来需要加注的地方。所有这些补加的注释都括在四角括号里,并且注有我的姓名的第一个字母或《D. H.》。\footnote{本卷括在花括号{}里,并注有弗·恩·。——编者注}

最近出版英文版时,曾对许多引文作了全面的校订,这是很必要的。马克思的小女儿爱琳娜不辞劳苦,对所有引文的原文都进行了核对,使占引文绝大多数的英文引文不再是德文的转译,而是它原来的英文原文。因此,在出第四版时,我必须参考这个恢复了原文的版本。在参考中发现了某些细小的不确切的地方:有的引文页码弄错了(这一部分是由于从笔记本上转抄时抄错了,一部分是由于前三版堆积下来的排印的错误);有的引号和省略号放错了位置(从札记本上抄录这么多的引文,这种差错是不可避免的);还有某些引文在翻译时用字不很恰当。有一些引文是根据马克思在1843—1845年在巴黎记的旧笔记本抄录的,当时马克思还不懂英语,他读英国经济学家的著作是读的法译本;那些经过两次转译的引文多少有些走了原意——如引自斯图亚特、尤尔等人著作的话就是如此。这些地方我都改以英文原文为根据。其他一些细小的不确切和疏忽的地方也都改正了。把第四版和以前各版对照一下,读者就会看出,所有这些细微的改正,并没有使本书的内容有丝毫值得一提的改变。只有一段引文没有找到出处,这就是理查·琼斯的一段话(第4版第562页注47\footnote{见本卷第656页。——编者注});多半是马克思把书名写错了[21]。所有其余的引文都仍然具有充分的说服力,甚至由于现在更加确切而更加具有说服力了。

不过,在此我不得不回溯一段往事。

据我所知,马克思的引文的正确性只有一次被人怀疑过。由于马克思逝世后这段引文的事又被重新提起,所以我不能不讲一讲。[22]

1872年3月7日,德国工厂主联盟的机关刊物柏林《协和》杂志刊登了一篇匿名作者的文章,标题是《卡尔·马克思是怎样引证的》。这篇文章的作者义愤填膺、粗暴无礼地指责马克思歪曲地引证了格莱斯顿1863年4月16日预算演说中的话(这句话引用在1864年国际工人协会成立宣言[23]中,并且在《资本论》第1卷第4版第617页即第3版第670—671页\footnote{见本卷第715页。——编者注}上再次引用)。这句话就是:“财富和实力这样令人陶醉的增长……完全限于有产阶级。”这篇文章的作者说,在《汉萨德》的(准官方的)速记记录中根本没有马克思引的这句话。“但是在格莱斯顿的演说中根本没有这句话。他在演说中说的和这句话正好相反。〈接着是黑体字〉\footnote{本卷引文中凡是尖括号〈 〉内的话或标点符号都是马克思或恩格斯加的。——译者注}\textbf{马克思在形式上和实质上增添了这句话!”}

马克思在5月接到了这一期《协和》杂志,他在6月1日的《人民国家报》上回答了这个匿名作者。由于当时他已记不起这一句话是引自哪一家报纸的报道,所以只得从两种英文出版物中举出意思完全相同的这句话,接着他引用了《泰晤士报》的报道。根据这一报道,格莱斯顿说:

\texttt{“从财富的观点来看,这个国家的状况就是这样。我应当承认,我几乎会怀着忧虑和悲痛的心情来看待财富和实力这样令人陶醉的增长,如果我相信,这种增长仅限于富裕阶级的话。这里完全没有注意到工人居民的状况。我刚刚描述的增长,亦即以我认为十分确切的材料为根据的增长,完全限于有产阶级”。}

可见,格莱斯顿在这里是说,如果事实如此,他将感到悲痛,而事实\textbf{确实}是:实力和财富这样令人陶醉的增长完全\textbf{限于}有产阶级;至于准官方的《汉萨德》,马克思接着说道:“格莱斯顿先生非常明智地从事后经过炮制的他的这篇演说中删掉了无疑会使他这位英国财政大臣声誉扫地的一句话;不过,这是英国常见的议会传统,而决不是小拉斯克尔反对倍倍尔的新发明[24]。”

这个匿名作者越来越恼怒了。他在自己的答复(7月4日《协和》杂志)中,抛开了所有第二手的材料,羞羞答答地暗示,按“惯例”只能根据速记记录引用议会演说;但接着他硬说,《泰晤士报》的报道(其中有这句“增添”的话)和《汉萨德》的报道(其中没有这句话)“在实质上完全一致”,还说什么《泰晤士报》的报道所包含的意思“同成立宣言中这个声名狼藉的地方正好相反”,然而这位先生却尽量避而不谈这样一个事实:除了这种所谓“正好相反”的意思外,还恰恰有那个“声名狼藉的地方”。不过,匿名作者自己也感到难于招架,只有玩弄新的花招才能自拔。他把自己那篇象上面所证明的通篇“无耻地撒谎”的文章,塞满了极其难听的骂人话,什么“恶意”,“不诚实”,“捏造的材料”,“那个捏造的引文”,“无耻地撒谎”,“完全是伪造的引文”,“这种伪造”,“简直无耻”,等等。同时他又设法暗地里使争论的问题转向新的方面,并预告要“在另一篇文章中说明,我们〈即这个“不会捏造的”匿名作者〉认为格莱斯顿的话包含什么意思”。好象他那无关紧要的见解还有点意义似的!这另一篇文章在7月11日的《协和》杂志上刊登出来了。

马克思在8月7日的《人民国家报》上又作了一次答辩,这次还引用了1863年4月17日的《晨星报》和《晨报》的有关的地方。根据这两家报纸的报道,格莱斯顿说,他会怀着忧虑……的心情来看待财富和实力令人陶醉的增长,如果他相信,增长只限于富裕阶级的话,而这种增长确实只\textbf{限于}占有财产的阶级;可见,在这两种报道中,也都一字不差地重复着所谓马克思“增添”的那句话。马克思接着把《泰晤士报》的字句同《汉萨德》的字句加以对比后再一次断定,第二天早上出版的三种互不相干的报纸在这一点上完全相同的报道,显而易见地证实了这句话的真实性,而这句话在根据某种“惯例”审查过的《汉萨德》中却没有,用马克思的话说,这是格莱斯顿“事后隐瞒了”。马克思最后声明,他没有时间再同匿名作者争辩,而匿名作者好象也觉得够了,至少马克思以后再没有收到《协和》杂志。

这个事件看来就此终结而被人遗忘了。诚然后来有一两次从一些同剑桥大学有来往的人那里传来一些神秘的谣言,说什么马克思在《资本论》里犯了写作上的大错,但无论怎样仔细追究,都得不到任何确实的结果。可是,1883年11月29日,即马克思逝世后八个月,《泰晤士报》上登载了一封剑桥三一学院的来信,署名是塞德莱·泰勒。这个搞最温和的合作运动的小人物在来信中完全出乎意外地使我们终于不仅弄清了剑桥的谣言,而且也弄清了《协和》杂志上的那个匿名作者。

这个三一学院的小人物写道:

\texttt{“使人特别惊异的是,}\textbf{布伦坦诺教授}\texttt{(当时在布勒斯劳,现在斯特拉斯堡任教)终于……揭露了在国际〈成立〉宣言中引用格莱斯顿演说时所怀的恶意。卡尔·马克思先生……曾企图为此进行辩护,但很快就被布伦坦诺巧妙的攻击打垮了,而他在垂死的挣扎中还敢于断言,格莱斯顿先生在1863年4月17日《泰晤士报》刊登他的演说原文之后,加工炮制了一份供《汉萨德》登载的演说记录,删掉了一句无疑会使他这位英国财政大臣声誉扫地的话。当布伦坦诺通过仔细地对比不同的文本,证明《泰晤士报》和《汉萨德》的报道彼此一致,绝对没有通过狡猾的断章取义而给格莱斯顿的话硬加上的那个意思时,马克思就借口没有时间而拒绝继续进行论战!”}

这就是全部事情的真相!布伦坦诺先生在《协和》杂志上发动的匿名攻击,在剑桥生产合作社的幻想小说中是多么辉煌!你看,这个德国工厂主联盟的圣乔治这样摆着架式,这样挺着剑[25],进行“巧妙的攻击”,而恶龙马克思“很快被打垮”,倒在他的脚下,“在垂死的挣扎中”断了气!

但这种阿里欧斯托式的全部战斗描写,只是为了掩盖我们这位圣乔治的诡计。他在这里再也不提什么“增添”,什么“伪造”,而只是说“狡猾的断章取义”了。整个问题完全转向另一个方面了,至于为什么要这样做,圣乔治和他的剑桥的卫士当然非常清楚。

爱琳娜·马克思在《今日》月刊(1884年2月)上对泰勒做了答辩——因为《泰晤士报》拒绝刊登她的文章。她首先把辩论归结到原来的这一点上:是不是马克思“增添”了这句话?塞德莱·泰勒先生回答说,在他看来,在马克思和布伦坦诺之间的争论中,

\texttt{“格莱斯顿先生的演说中是否有这句话完全是次要问题,更主要的是,引用这句话的目的是正确传达格莱斯顿的意思,还是歪曲他的意思”。}

接着,他承认说,《泰晤士报》的报道“的确包含有文字上的矛盾”,但是,如果正确地推断,也就是照自由主义的格莱斯顿的意思推断,据说整个上下文正好表明了格莱斯顿所想说的那个意思(1884年3月《今日》月刊)。这里最可笑的是,虽然照匿名的布伦坦诺所说,按“惯例”应当从《汉萨德》引证,《泰晤士报》的报道“必然很粗糙”,但我们这个剑桥的小人物却固执地\textbf{不}从《汉萨德》引证,而从《泰晤士报》引证。当然,《汉萨德》上根本\textbf{没有}这句倒霉的话!

爱琳娜·马克思没有费很大力气就在同一期《今日》月刊上驳倒了这个论据。要么泰勒先生读过1872年的论战文章,如果是这样,那他现在就是在“撒谎”,他的撒谎表现在:他不但“增添”了原来没有的东西,而且“否定”了原来已有的东西。要么他根本没有读过这些论战文章,那他就根本无权开口。无论如何,他再也不敢支持他的朋友布伦坦诺控告马克思“增添”引文了。相反,现在他不是控告马克思“增添”,而是控告马克思删掉了一句重要的话。其实这句话被引用在成立宣言的第5页上,只在这句所谓“增添”的话上面几行。至于格莱斯顿演说中包含的“矛盾”,恰好正是马克思指出了(《资本论》第618页注105\footnote{见本卷第716页。——编者注},即第3版第672页)“1863年和1864年格莱斯顿的预算演说中不断出现的显著的矛盾”!不过,他不象塞德莱·泰勒那样企图把这些矛盾溶化在自由主义的温情之中。爱·马克思在答辩的结尾说:“事实上完全相反。马克思既没有删掉任何值得一提的东西,也绝对没有‘增添’任何东西。他只是把格莱斯顿在演说中确实说过、而又用某种方法从《汉萨德》的报道中抹掉的一句话重新恢复,使它不致被人们遗忘。”

从此以后,连塞德莱·泰勒先生也闭口不言了。大学教授们所发动的整个这场攻击,在两大国持续二十年之久,而其结果是任何人也不敢再怀疑马克思写作上的认真态度了。可以想象得到,正如布伦坦诺先生不会再相信《汉萨德》象教皇般永无谬误那样,塞德莱·泰勒先生今后也将不会再相信布伦坦诺先生的文坛战报了。



\begin{flushright}
    \textbf{弗里德思希·恩格斯}\\
    \small{1890年6月25日于伦敦}
\end{flushright}






%————————————
% 章节

\mainmatter

%第一篇
\part{商品和货币}
\thispagestyle{empty}
%\setcounter{page}{1}
%\pagenumbering{arabic}

%第一章
\chapter{商品}

    \section{商品的两个因素:使用价值和价值(价值实体,价值量)}

    资本主义生产方式占统治地位的社会的财富,表现为“庞大的商品堆积”\footnote{卡尔·马克思《政治经济学批判》1859年柏林版第3页[26]。},单个的商品表现为这种财富的元素形式。因此,我们的研究就从分析商品开始。

    商品首先是一个外界的对象,一个靠自己的属性来满足人的某种需要的物。这种需要的性质如何,例如是由胃产生还是由幻想产生,是与问题无关的。\footnote{“欲望包含着需要;这是精神的食欲,就象肉体的饥饿那样自然……大部分〈物〉具有价值,是因为它们满足精神的需要。”(尼古拉·巴尔本《新币轻铸论。答洛克先生关于提高货币价值的意见》1696年伦敦版第2、3页)}这里的问题也不在于物怎样来满足人的需要,是作为生活资料即消费品来直接满足,还是作为生产资料来间接满足。

    每一种有用物,如铁、纸等等,都可以从质和量两个角度来考察。每一种这样的物都是许多属性的总和,因此可以在不同的方面有用。发现这些不同的方面,从而发现物的多种使用方式,是历史的事情。\footnote{“物都有内在的长处〈这是巴尔本用来表示使用价值的专门用语〉,这种长处在任何地方都是一样的,如磁石吸铁的长处就是如此。”(尼古拉·巴尔本《新币轻铸论。答洛克先生关于提高货币价值的意见》1696年伦敦版第6页)磁石吸铁的属性只是在通过它发现了磁极性以后才成为有用的。}为有用物的量找到社会尺度,也是这样。商品尺度之所以不同,部分是由于被计量的物的性质不同,部分是由于约定俗成。

    物的有用性使物成为使用价值。\footnote{“任何物的自然worth〔价值〕都在于它能满足必要的需要,或者给人类生活带来方便。”(约翰·洛克《论降低利息的后果》(1691年),载于《约翰·洛克著作集》1777年伦敦版第2卷第28页)在十七世纪,我们还常常看到英国著作家用《worth》表示使用价值,用《value》表示交换价值;这完全符合英语的精神,英语喜欢用日耳曼语源的词表示直接的东西,用罗马语源的词表示被反射的东西。}[但这种有用性不是悬在空中的。它决定于商品体的属性,离开了商品体就不存在。因此,商品体本身,例如铁、小麦、金钢石等等,就是使用价值,或财物。商品体的这种性质,同人取得它的使用属性所耗费的劳动的多少没有关系。在考察使用价值时,总是以它们有一定的量为前提,如几打表,几码布,几吨铁等等。商品的使用价值为商品学这门学科提供材料。\footnote{在资产阶级社会中,流行着一种法律上的假定,认为每个人作为商品的买者都具有百科全书般的商品知识。}使用价值只是在使用或消费中得到实现。不论财富的社会形式如何,使用价值总是构成财富的物质内容。在我们所要考察的社会形式中,使用价值同时又是交换价值的物质承担者。

    交换价值首先表现为一种使用价值同另一种使用价值相交换的量的关系或比例\footnote{“价值就是一物和另一物、一定量的这种产品和一定量的别种产品之间的交换关系。”(列特隆《论社会利益》,[载于]【本卷中凡是四角括号[ ]内的话都是德文版编者加的。——译者注】德尔编《重农学派》1846年巴黎版第889页)},这个比例随着时间和地点的不同而不断改变。因此,交换价值好象是一种偶然的、纯粹相对的东西,也就是说,商品固有的、内在的交换价值似乎是一个形容语的矛盾\footnote{形容语的矛盾的原文是《contradictio in adjecto》,指“圆形的方”,“木制的铁”一类的矛盾。——译者注}。\footnote{“任何东西都不可能有内在的交换价值。”(尼·巴尔本《新币轻铸论。答洛克先生关于提高货币价值的意见》第6页)或者象巴特勒所说: %!译者注待区分*2   +++++++++++++++++++++++++++++++++

    “物的价值正好和它会换来的东西相等。”[27]}现在我们进一步考察这个问题。

    某种一定量的商品,例如一夸特小麦,同x量鞋油或y量绸缎或z量金等等交换,总之,按各种极不相同的比例同别的商品交换。因此,小麦有许多种交换价值,而不是只有一种。既然x量鞋油、y量绸缎、z量金等等都是一夸特小麦的交换价值,那末,x量鞋油、y量绸缎、z量金等等就必定是能够互相代替的或同样大的交换价值。由此可见,第一,同一种商品的各种有效的交换价值表示一个等同的东西。第二,交换价值只能是可以与它相区别的某种内容的表现方式,“表现形式”。

    我们再拿两种商品例如小麦和铁来说。不管二者的交换比例怎样,总是可以用一个等式来表示:一定量的小麦等于若干量的铁,如1夸特小麦=a吨铁。这个等式说明什么呢?它说明在两种不同的物里面,即在1夸特小麦和a吨铁里面,有一种等量的共同的东西。因而这二者都等于第三种东西,后者本身既不是第一种物,也不是第二种物。这样,二者中的每一个只要是交换价值,就必定能化为这第三种东西。

    用一个简单的几何学例子就可以说明这一点。为了确定和比较各种直线形的面积,就把它们分成三角形,再把三角形化成与它的外形完全不同的表现——底乘高的一半。各种商品的交换价值也同样要化成一种共同东西,各自代表这种共同东西的多量或少量。

    这种共同东西不可能是商品的几何的、物理的、化学的或其他的天然属性。商品的物体属性只是就它们使商品有用,从而使商品成为使用价值来说,才加以考虑。另一方面,商品交换关系的明显特点,正在于抽去商品的使用价值。在商品交换关系中,只要比例适当,一种使用价值就和其他任何一种使用价值完全相等。或者象老巴尔本说的:

    “只要交换价值相等,一种商品就同另一种商品一样。交换价值相等的物是没有任何差别或区别的。”\footnote{“只要交换价值相等,一种商品就同另一种商品一样。交换价值相等的物是没有任何差别或区别的……价值100镑的铅或铁与价值100镑的银和金具有相等的交换价值。”(尼·巴尔本《新币轻铸论。答洛克先生关于提高货币价值的意见》第53页和第7页)}

    作为使用价值,商品首先有质的差别;作为交换价值,商品只能有量的差别,因而不包含任何一个使用价值的原子。

    如果把商品体的使用价值撇开,商品体就只剩下一个属性,即劳动产品这个属性。可是劳动产品在我们手里也已经起了变化。如果我们把劳动产品的使用价值抽去,那末也就是把那些使劳动产品成为使用价值的物质组成部分和形式抽去。它们不再是桌子、房屋、纱或别的什么有用物。它们的一切可以感觉到的属性都消失了。它们也不再是木匠劳动、瓦匠劳动、纺纱劳动,或其他某种一定的生产劳动的产品了。随着劳动产品的有用性质的消失,体现在劳动产品中的各种劳动的有用性质也消失了,因而这些劳动的各种具体形式也消失了。各种劳动不再有什么差别,全都化为相同的人类劳动,抽象人类劳动。

    现在我们来考察劳动产品剩下来的东西。它们剩下的只是同一的幽灵般的对象性\footnote{对象性的原文是《Gegenständlichkeit》,意思是:客观现实性,客观存在的东西。——译者注},只是无差别的人类劳动的单纯凝结,即不管以哪种形式进行的人类劳动力耗费的单纯凝结。这些物现在只是表示,在它们的生产上耗费了人类劳动力,积累了人类劳动。这些物,作为它们共有的这个社会实体的结晶,就是价值——商品价值。

    我们已经看到,在商品的交换关系本身中,商品的交换价值表现为同它们的使用价值完全无关的东西。如果真正把劳动产品的使用价值抽去,就得到刚才已经规定的它们的价值。因此,在商品的交换关系或交换价值中表现出来的共同东西,也就是商品的价值。研究的进程会使我们再把交换价值当作价值的必然的表现方式或表现形式来考察,但现在,我们应该首先不管这种形式来考察价值。

    可见,使用价值或财物具有价值,只是因为有抽象人类劳动体现或物化在里面。那末,它的价值量是怎样计量的呢?是用它所包含的“形成价值的实体”即劳动的量来计量。劳动本身的量是用劳动的持续时间来计量,而劳动时间又是用一定的时间单位如小时、日等作尺度。

    可能会有人这样认为,既然商品的价值由生产商品所耗费的劳动量来决定,那末一个人越懒,越不熟练,他的商品就越有价值,因为他制造商品需要花费的时间越多。但是,形成价值实体的劳动是相同的人类劳动,是同一的人类劳动力的耗费。体现在商品世界全部价值中的社会的全部劳动力,在这里是当作一个同一的人类劳动力,虽然它是由无数单个劳动力构成的。每一个这种单个劳动力,同别一个劳动力一样,都是同一的人类劳动力,只要它具有社会平均劳动力的性质,起着这种社会平均劳动力的作用,从而在商品的生产上只使用平均必要劳动时间或社会必要劳动时间。社会必要劳动时间是在现有的社会正常的生产条件下,在社会平均的劳动熟练程度和劳动强度下制造某种使用价值所需要的劳动时间。例如,在英国采用蒸汽织布机以后,把一定量的纱织成布所需要的劳动可能比过去少一半。实际上,英国的手工织布工人把纱织成布仍旧要用以前那样多的劳动时间,但这时他一小时的个人劳动的产品只代表半小时的社会劳动,因此价值也降到了它以前的一半。

    因此,如果生产商品所需要的劳动时间不变,商品的价值量也就不变。但是,生产商品所需要的劳动时间随着劳动生产力的每一变动而变动。劳动生产力是由多种情况决定的,其中包括:工人的平均熟练程度,科学的发展水平和它在工艺上应用的程度,生产过程的社会结合,生产资料的规模和效能,以及自然条件。例如,同一劳动量在丰收年表现为8蒲式耳小麦,在歉收年只表现为4蒲式耳。同一劳动量用在富矿比用在贫矿能提供更多的金属等等。金刚石在地壳中是很稀少的,因而发现金刚石平均要花很多劳动时间。因此,很小一块金刚石就代表很多劳动。杰科布曾经怀疑金是否按其全部价值支付过。[29]至于金刚石,就更可以这样说了。厄什韦葛说过,到1823年,巴西金刚石矿八十年的总产量的价格还赶不上巴西甘蔗种植园或咖啡种植园一年半平均产量的价格,虽然前者代表的劳动多得多,从而价值也多得多。如果发现富矿,同一劳动量就会表现为更多的金刚石,而金刚石的价值就会降低。假如能用不多的劳动把煤变成金刚石,金刚石的价值就会低于砖的价值。总之,劳动生产力越高,生产一种物品所需要的劳动时间就越少,凝结在该物品中的劳动量就越小,该物品的价值就越小。相反地,劳动生产力越低,生产一种物品的必要劳动时间就越多,该物品的价值就越大。可见,商品的价值量与体现在商品中的劳动的量成正比,与这一劳动的生产力成反比。\footnote{在第一版中接着有这样一段话:我们现在知道了价值的实体。这就是劳动。我们知道了价值的量的尺度。这就是劳动时间。价值的形式(正是它使价值成为交换价值),有待分析。现在先要较详细地阐明那些已经发现的规定。——编者注}

    一个物可以是使用价值而不是价值。在这个物并不是由于劳动而对人有用的情况下就是这样。例如,空气、处女地、天然草地、野生林等等。一个物可以有用,而且是人类劳动产品,但不是商品。谁用自己的产品来满足自己的需要,他生产的就只是使用价值,而不是商品。要生产商品,他不仅要生产使用价值,而且要为别人生产使用价值,即生产社会的使用价值。{而且不只是单纯为别人。中世纪农民为封建主生产交代役租的粮食,为神父生产纳什一税的粮食。但不管是交代役租的粮食,还是纳什一税的粮食,都并不因为是为别人生产的,就成为商品。要成为商品,产品必须通过交换,转到把它当作使用价值使用的人的手里。}\footnote{第4版注:我插进了括号里的这段话,因为省去这段话常常会引起误解,好象不是由生产者本人消费的产品,马克思都认为是商品。——弗·恩·}最后,没有一个物可以是价值而不是使用物品。如果物没有用,那末其中包含的劳动也就没有用,不能算作劳动,因此不形成价值。

    \section{体现在商品中的劳动的二重性}

    起初我们看到,商品是一种二重的东西,即使用价值和交换价值。后来表明,劳动就它表现为价值而论,也不再具有它作为使用价值的创造者所具有的那些特征。商品中包含的劳动的这种二重性,是首先由我批判地证明了的。\footnote{卡尔·马克思《政治经济学批判》1859年柏林版第12、13等页[30]。}这一点是理解政治经济学的枢纽,因此,在这里要较详细地加以说明。

    我们就拿两种商品如1件上衣和10码麻布来说。假定前者的价值比后者的价值大一倍。假设10码麻布=W,则1件上衣=2W。

    上衣是满足一种特殊需要的使用价值。要生产上衣,就需要进行特定种类的生产活动。这种生产活动是由它的目的、操作方式、对象、手段和结果决定的。由自己产品的使用价值或者由自己产品是使用价值来表示自己的有用性的劳动,我们简称为有用劳动。从这个观点来看,劳动总是联系到它的有用效果来考察的。

    上衣和麻布是不同质的使用价值,同样,决定它们存在的劳动即缝和织,也是不同质的。如果这些物不是不同质的使用价值,从而不是不同质的有用劳动的产品,它们就根本不能作为商品来互相对立。上衣不会与上衣交换,一种使用价值不会与同种的使用价值交换。

    各种使用价值或商品体的总和,表现了同样多种的、按照属、种、科、亚种、变种分类的有用劳动的总和,即表现了社会分工。这种分工是商品生产存在的条件,虽然不能反过来说商品生产是社会分工存在的条件。在古代印度公社中就有社会分工,但产品并不成为商品。或者拿一个较近的例子来说,每个工厂内都有系统的分工,但是这种分工不是通过工人交换他们个人的产品来实现的。只有独立的互不依赖的私人劳动的产品,才作为商品互相对立。

    可见,每个商品的使用价值都包含着一定的有目的的生产活动,或有用劳动。各种使用价值如果不包含不同质的有用劳动,就不能作为商品互相对立。在产品普遍采取商品形式的社会里,也就是在商品生产者的社会里,作为独立生产者的私事而各自独立进行的各种有用劳动的这种质的区别,发展成一个多支的体系,发展成社会分工。

    对上衣来说,无论是裁缝自己穿还是他的顾客穿,都是一样的。在这两种场合,它都是起使用价值的作用。同样,上衣和生产上衣的劳动之间的关系,也并不因为裁缝劳动成为专门职业,成为社会分工的一个独立的部分就有所改变。在有穿衣需要的地方,在有人当裁缝以前,人已经缝了几千年的衣服。但是,上衣、麻布以及任何一种不是天然存在的物质财富要素,总是必须通过某种专门的、使特殊的自然物质适合于特殊的人类需要的、有目的的生产活动创造出来。因此,劳动作为使用价值的创造者,作为有用劳动,是不以一切社会形式为转移的人类生存条件,是人和自然之间的物质变换即人类生活得以实现的永恒的自然必然性。

    上衣、麻布等等使用价值,简言之,种种商品体,是自然物质和劳动这两种要素的结合。如果把上衣、麻布等等包含的各种不同的有用劳动的总和除外,总还剩有一种不借人力而天然存在的物质基质。人在生产中只能象自然本身那样发挥作用,就是说,只能改变物质的形态。\footnote{“宇宙的一切现象,不论是由人手创造的,还是由物理学的一般规律引起的,都不是真正的新创造,而只是物质的形态变化。结合和分离是人的智慧在分析再生产的观念时一再发现的唯一要素;价值〈指使用价值,尽管维里在这里同重农学派论战时自己也不清楚说的是哪一种价值〉和财富的再生产,如土地、空气和水在田地上变成谷物,或者昆虫的分泌物经过人的手变成丝绸,或者一些金属片被装配成钟表,也是这样。”(彼得罗·维里《政治经济学研究》1771年初版,载于库斯托第编《意大利政治经济学名家文集》现代部分,第15卷第21、22页)}不仅如此,他在这种改变形态的劳动中还要经常依靠自然力的帮助。因此,劳动并不是它所生产的使用价值即物质财富的唯一源泉。正象威廉·配第所说,劳动是财富之父,土地是财富之母。[31]

    现在,我们放下作为使用物品的商品,来考察商品价值。

    我们曾假定,上衣的价值比麻布大一倍。但这只是量的差别,我们先不去管它。我们要记住的是,假如1件上衣的价值比10码麻布的价值大一倍,那末,20码麻布就与1件上衣具有同样的价值量。作为价值,上衣和麻布是有相同实体的物,是同种劳动的客观表现。但缝和织是不同质的劳动。然而在有些社会状态下,同一个人时而缝时而织,因此,这两种不同的劳动方式只是同一个人的劳动的变化,还不是不同的人的专门固定职能,正如我们的裁缝今天缝上衣和明天缝裤子只是同一个人的劳动的变化一样。其次,一看就知道,在我们资本主义社会里,随着劳动需求方向的改变,总有一定部分的人类劳动时而采取缝的形式,时而采取织的形式。劳动形式发生这种变换时不可能没有摩擦,但这种变换是必定要发生的。如果把生产活动的特定性质撇开,从而把劳动的有用性质撇开,生产活动就只剩下一点:它是人类劳动力的耗费。尽管缝和织是不同质的生产活动,但二者都是人的脑、肌肉、神经、手等等的生产耗费,从这个意义上说,二者都是人类劳动。这只是耗费人类劳动力的两种不同的形式。当然,人类劳动力本身必须已有一定的发展,才能以这种或那种形式耗费。但是,商品价值体现的是人类劳动本身,是一般人类劳动的耗费。正如在资产阶级社会里,将军或银行家扮演着重要的角色,而人本身则扮演极卑微的角色一样\footnote{参看黑格尔《法哲学》1840年柏林版第250页第190节。},人类劳动在这里也是这样。它是每个没有任何专长的普通人的机体平均具有的简单劳动力的耗费。简单平均劳动虽然在不同的国家和不同的文化时代具有不同的性质,但在一定的社会里是一定的。比较复杂的劳动只是自乘的或不如说多倍的简单劳动,因此,少量的复杂劳动等于多量的简单劳动。经验证明,这种简化是经常进行的。一个商品可能是最复杂的劳动的产品,但是它的价值使它与简单劳动的产品相等,因而本身只表示一定量的简单劳动。\footnote{读者应当注意,这里指的不是工人得到的一个工作日的工资或价值,而是指工人的一个工作日物化成的商品价值。在我们叙述的这个阶段,工资这个范畴根本还不存在。}各种劳动化为当作它们的计量单位的简单劳动的不同比例,是在生产者背后由社会过程决定的,因而在他们看来,似乎是由习惯确定的。为了简便起见,我们以后把各种劳动力直接当作简单劳动力,这样就省去了简化的麻烦。

    因此,正如在作为价值的上衣和麻布中,它们的使用价值的差别被抽去一样,在表现为这些价值的劳动中,劳动的有用形式即缝和织的区别也被抽去了。作为使用价值的上衣和麻布是有一定目的的生产活动同布和纱的结合,而作为价值的上衣和麻布,不过是同种劳动的凝结,同样,这些价值所包含的劳动之所以算作劳动,并不是因为它们同布和纱发生了生产的关系,而只是因为它们是人类劳动力的耗费。正是由于缝和织具有不同的质,它们才是形成作为使用价值的上衣和麻布的要素;而只是由于它们的特殊的质被抽去,由于它们具有相同的质,即人类劳动的质,它们才是上衣价值和麻布价值的实体。

    可是,上衣和麻布不仅是价值,而且是一定量的价值。我们曾假定,1件上衣的价值比10码麻布的价值大一倍。它们价值量的这种差别是从哪里来的呢?这是由于麻布包含的劳动只有上衣的一半,因而生产后者所要耗费劳动力的时间必须比生产前者多一倍。

    因此,就使用价值说,有意义的只是商品中包含的劳动的质,就价值量说,有意义的只是商品中包含的劳动的量,不过这种劳动已经化为没有质的区别的人类劳动。在前一种情况下,是怎样劳动,什么劳动的问题;在后一种情况下,是劳动多少,劳动时间多长的问题。既然商品的价值量只是表示商品中包含的劳动量,那末,在一定的比例上,各种商品应该总是等量的价值。

    如果生产一件上衣所需要的一切有用劳动的生产力不变,上衣的价值量就同上衣的数量一起增加。如果一件上衣代表x个工作日,两件上衣就代表2x个工作日,依此类推。假定生产一件上衣的必要劳动增加一倍或减少一半。在前一种场合,一件上衣就具有以前两件上衣的价值,在后一种场合,两件上衣就只有以前一件上衣的价值,虽然在这两种场合,上衣的效用和从前一样,上衣包含的有用劳动的质也和从前一样。但生产上衣所耗费的劳动量有了变化。

    更多的使用价值本身就是更多的物质财富,两件上衣比一件上衣多。两件上衣可以两个人穿,一件上衣只能一个人穿,依此类推。然而随着物质财富的量的增长,它的价值量可能同时下降。这种对立的运动来源于劳动的二重性。生产力当然始终是有用的具体的劳动的生产力,它事实上只决定有目的的生产活动在一定时间内的效率。因此,有用劳动成为较富或较贫的产品源泉与有用劳动的生产力的提高或降低成正比。相反地,生产力的变化本身丝毫也不会影响表现为价值的劳动。既然生产力属于劳动的具体有用形式,它自然不再同抽去了具体有用形式的劳动有关。因此,不管生产力发生了什么变化,同一劳动在同样的时间内提供的价值量总是相同的。但它在同样的时间内提供的使用价值量会是不同的:生产力提高时就多些,生产力降低时就少些。因此,那种能提高劳动成效从而增加劳动所提供的使用价值量的生产力变化,如果会缩减生产这个使用价值量所必需的劳动时间的总和,就会减少这个增大的总量的价值量。反之亦然。

    一切劳动,从一方面看,是人类劳动力在生理学意义上的耗费;作为相同的或抽象的人类劳动,它形成商品价值。一切劳动,从另一方面看,是人类劳动力在特殊的有一定目的的形式上的耗费;作为具体的有用劳动,它生产使用价值。\footnote{第2版注:为了证明“只有劳动才是我们在任何时候都能够用来估计和比较各种商品价值的最后的和现实的唯一尺度”,亚·斯密写道:“等量的劳动在任何时候和任何地方对工人本身都必定具有同样的价值。在工人的健康、精力和活动正常的情况下,在他所能具有的平均熟练程度的情况下,他总是要牺牲同样多的安宁、自由和幸福”(《国富论》第1卷第5章[第104—105页])。一方面,亚·斯密在这里(不是在每一处)把价值决定于生产商品所耗费的劳动量,同商品价值决定于劳动的价值混为一谈,因而他力图证明,等量的劳动总是具有同样的价值。另一方面,他感觉到,劳动就它表现为商品的价值而论,只是劳动力的耗费,但他把这种耗费又仅仅理解为牺牲安宁、自由和幸福,而不是把它也看作正常的生命活动。诚然,他看到的是现代雇佣工人。——注(9)提到的亚·斯密的那位匿名的前辈的说法要恰当得多。他说:“某人制造这种必需品用了一个星期……而拿另一种物与他进行交换的人要确切地估计出什么是真正的等值物,最好计算出什么东西会花费自己同样多的labour〔劳动〕和时间。这实际上就是说:一个人在一定时间内在一物上用去的劳动,同另一个人在同样的时间内在另一物上用去的劳动相交换。”(《对货币利息,特别是公债利息的一些看法》第39页)——{第4版注:英语有一个优点,它有两个不同的词来表达劳动的这两个不同的方面。创造使用价值的并具有一定质的劳动叫做Work,以与labour相对;创造价值并且只在量上被计算的劳动叫做labour,以与Work相对。见英译本第14页脚注。——弗·恩·}}

    \section{价值形式或交换价值}

    商品是以铁、麻布、小麦等等使用价值或商品体的形式出现的。这是它们的日常的自然形式。但它们所以是商品,只因为它们是二重物,既是使用物品又是价值承担者。因此,它们表现为商品或具有商品的形式,只是由于它们具有二重的形式,即自然形式和价值形式。

    商品的价值对象性不同于快嘴桂嫂,你不知道对它怎么办。[32]同商品体的可感觉的粗糙的对象性正好相反,在商品体的价值对象性中连一个自然物质原子也没有。因此,每一个商品不管你怎样颠来倒去,它作为价值物总是不可捉摸的。但是如果我们记住,商品只有作为同一的社会单位即人类劳动的表现才具有价值对象性,因而它们的价值对象性纯粹是社会的,那末不用说,价值对象性只能在商品同商品的社会关系中表现出来。我们实际上也是从商品的交换价值或交换关系出发,才探索到隐藏在其中的商品价值。现在我们必须回到价值的这种表现形式。

    谁都知道——即使他别的什么都不知道,——商品具有同它们使用价值的五光十色的自然形式成鲜明对照的、共同的价值形式,即货币形式。但是在这里,我们要做资产阶级经济学从来没有打算做的事情:指明这种货币形式的起源,就是说,探讨商品价值关系中包含的价值表现,怎样从最简单的最不显眼的样子一直发展到炫目的货币形式。这样,货币的谜就会随着消失。

    显然,最简单的价值关系就是一个商品同另一个不同种的商品(不管是哪一种商品都一样)的价值关系。因此,两个商品的价值关系为一个商品提供了最简单的价值表现。

        \subsection{简单的、个别的或偶然的价值形式}

        \begin{center}
            x量商品A=y量商品B,或x量商品A值y量商品B。\\
            (20码麻布=1件上衣,或20码麻布值1件上衣。)
        \end{center}
        
            \subsubsection{价值表现的两极:相对价值形式和等价形式}

            一切价值形式的秘密都隐藏在这个简单的价值形式中。因此,分析这个形式确实困难。
            
            两个不同种的商品A和B,如我们例子中的麻布和上衣,在这里显然起着两种不同的作用。麻布通过上衣表现自己的价值,上衣则成为这种价值表现的材料。前一个商品起主动作用,后一个商品起被动作用。前一个商品的价值表现为相对价值,或者说,处于相对价值形式。后一个商品起等价物的作用,或者说,处于等价形式。
            
            相对价值形式和等价形式是同一价值表现的互相依赖、互为条件、不可分离的两个要素,同时又是同一价值表现的互相排斥、互相对立的两端即两极;这两种形式总是分配在通过价值表现互相发生关系的不同的商品上。例如我不能用麻布来表现麻布的价值。20码麻布=20码麻布,这不是价值表现。相反,这个等式只是说,20码麻布无非是20码麻布,是一定量的使用物品麻布。因此,麻布的价值只能相对地表现出来,即通过另一个商品表现出来。因此,麻布的相对价值形式要求有另一个与麻布相对立的商品处于等价形式。另一方面,这另一个充当等价物的商品不能同时处于相对价值形式。它不表现自己的价值。它只是为别一个商品的价值表现提供材料。
            
            诚然,20码麻布=1件上衣,或20码麻布值1件上衣,这种表现也包含着相反的关系:1件上衣=20码麻布,或1件上衣值20码麻布。但是,要相对地表现上衣的价值,我就必须把等式倒过来,而一旦我这样做,成为等价物的就是麻布,而不是上衣了。可见,同一个商品在同一个价值表现中,不能同时具有两种形式。不仅如此,这两种形式是作为两极互相排斥的。
            
            一个商品究竟是处于相对价值形式,还是处于与之对立的等价形式,完全取决于它当时在价值表现中所处的地位,就是说,取决于它是价值被表现的商品,还是表现价值的商品。

            \subsubsection{相对价值形式}

                \paragraph{相对价值形式的内容}

                要发现一个商品的简单价值表现怎样隐藏在两个商品的价值关系中,首先必须完全撇开这个价值关系的量的方面来考察这个关系。人们通常的做法正好相反,他们在价值关系中只看到两种商品的一定量彼此相等的比例。他们忽略了,不同物的量只有化为同一单位后,才能在量上互相比较。不同物的量只有作为同一单位的表现,才是同名称的,因而是可通约的。\footnote{少数经济学家,例如赛·贝利,曾分析价值形式,但没有得到任何结果,这首先是因为他们把价值形式同价值混为一谈,其次,是因为在讲求实用的资产者的粗鄙的影响下,他们一开始就只注意量的规定性。“对量的支配……构成价值。”(《货币及其价值的变动》1837年伦敦版第11页)作者赛·贝利。}
                
                不论20码麻布=1件上衣,或=20件上衣,或=x件上衣,也就是说,不论一定量的麻布值多少件上衣,每一个这样的比例总是包含这样的意思:麻布和上衣作为价值量是同一单位的表现,是同一性质的物。麻布=上衣是这一等式的基础。
                
                但是,这两个被看作质上等同的商品所起的作用是不同的。只有麻布的价值得到表现。是怎样表现的呢?是通过同上衣的关系,把上衣当作它的“等价物”,或与它“能交换的东西”。在这个关系中,上衣是价值的存在形式,是价值物,因为只有作为价值物,它才是与麻布相同的。另一方面,麻布自身的价值显示出来了,或得到了独立的表现,因为麻布只有作为价值才能把上衣当作等值的东西,或与它能交换的东西。比如,丁酸是同甲酸丙酯不同的物体。但二者是由同一些化学实体——碳(C)、氢(H)、氧(O)构成,而且是以相同的百分比构成,即C4H8O2。假如甲酸丙酯被看作与丁酸相等,那末,在这个关系中,第一,甲酸丙酯只是C4H8O2的存在形式,第二,就是说,丁酸也是由C4H8O2构成的。可见,通过使甲酸丙酯同丁酸相等,丁酸与自身的物体形态不同的化学实体被表现出来了。
                
                如果我们说,商品作为价值只是人类劳动的凝结,那末,我们的分析就是把商品化为价值抽象,但是并没有使它们具有与它们的自然形式不同的价值形式。在一个商品和另一个商品的价值关系中,情形就不是这样。在这里,一个商品的价值性质通过该商品与另一个商品的关系而显露出来。
                
                例如当上衣作为价值物被看作与麻布相等时,前者包含的劳动就被看作与后者包含的劳动相等。固然,缝上衣的劳动是一种与织麻布的劳动不同的具体劳动。但是,把缝看作与织相等,实际上就是把缝化为两种劳动中确实等同的东西,化为它们的人类劳动的共同性质。通过这种间接的办法还说明,织就它织出价值而论,也和缝毫无区别,所以是抽象人类劳动。只有不同种商品的等价表现才使形成价值的劳动的这种特殊性质显示出来,因为这种等价表现实际上是把不同种商品所包含的不同种劳动化为它们的共同东西,化为一般人类劳动。\footnote{第2版注:最早的经济学家之一、著名的富兰克林,继威廉·配第之后看出了价值的本质,他说:“既然贸易无非是一种劳动同另一种劳动的交换,所以一切物的价值用劳动来估计是最正确的”(斯巴克斯编《富兰克林全集》1836年波士顿版第2卷第267页)。富兰克林没有意识到,既然他“用劳动”来估计一切物的价值,也就抽掉了各种互相交换的劳动的差别,这样就把这些劳动化为相同的人类劳动。他虽然没有意识到这一点,却把它说了出来。他先说“一种劳动”,然后说“另一种劳动”,最后说的是没有任何限定的“劳动”,也就是作为一切物的价值实体的劳动。}
                
                然而,只把构成麻布价值的劳动的特殊性质表现出来,是不够的。处于流动状态的人类劳动力或人类劳动形成价值,但本身不是价值。它在凝固的状态中,在物化的形式上才成为价值。要使麻布的价值表现为人类劳动的凝结,就必须使它表现为一种“对象性”,这种对象性与麻布本身的物体不同,同时又是麻布与其他商品所共有的。这个问题已经解决了。
                
                在麻布的价值关系中,上衣是当作与麻布同质的东西,是当作同一性质的物,因为它是价值。在这里,它是当作表现价值的物,或者说,是以自己的可以捉摸的自然形式表示价值的物。当然,上衣,作为商品体的上衣,只是使用价值。一件上衣同任何一块麻布一样,不表现价值。这只是证明,上衣在同麻布的价值关系中,比在这种关系之外,多一层意义,正象许多人穿上镶金边的上衣,比不穿这种上衣,多一层意义一样。
                
                在上衣的生产上,人类劳动力的确是以缝的形式被耗费的。因此,上衣中积累了人类劳动。从这方面看,上衣是“价值承担者”,虽然它的这种属性即使把它穿破了也是看不出来的。在麻布的价值关系中,上衣只是显示出这一方面,也就是当作物体化的价值,当作价值体。即使上衣扣上了纽扣,麻布在它身上还是认出与自己同宗族的美丽的价值灵魂。但是,如果对麻布来说,价值不同时采取上衣的形式,上衣在麻布面前就不能表示价值。例如,如果在A看来,陛下不具有B的仪表,因而不随着国王的每次更换而改变容貌、头发等等,A就不会把B当作陛下。
                
                可见,在上衣成为麻布的等价物的价值关系中,上衣形式起着价值形式的作用。因此,商品麻布的价值是表现在商品上衣的物体上,一个商品的价值表现在另一个商品的使用价值上。作为使用价值,麻布是在感觉上与上衣不同的物;作为价值,它却是“与上衣等同的东西”,因而看起来就象上衣。麻布就这样取得了与它的自然形式不同的价值形式。它的价值性质通过它和上衣相等表现出来,正象基督徒的羊性通过他和上帝的羔羊相等表现出来一样。
                
                我们看到,一当麻布与别的商品即上衣交往时,商品价值的分析向我们说明的一切,现在就由麻布自己说出来了。不过它只能用它自己通晓的语言即商品语言来表达它的思想。为了说明劳动在人类劳动的抽象属性上形成它自己的价值,它就说,上衣只要与它相等,从而是价值,就和麻布一样是由同一劳动构成的。为了说明它的高尚的价值对象性不同于它的浆硬的物体,它就说,价值看起来象上衣,因此它自己作为价值物,就同上衣相象,正如两个鸡蛋相象一样。顺便指出,除希伯来语以外,商品语言中也还有其他许多确切程度不同的方言。例如,要表达商品B同商品A相等是商品A自己的价值表现,德文《Wertsein》〔价值,价值存在〕就不如罗曼语的动词valere,valer,valoir〔值〕表达得确切。巴黎确实值一次弥撒![33]
                
                可见,通过价值关系,商品B的自然形式成了商品A的价值形式,或者说,商品B的物体成了反映商品A的价值的镜子。\footnote{在某种意义上,人很象商品。因为人来到世间,既没有带着镜子,也不象费希特派的哲学家那样,说什么我就是我,所以人起初是以别人来反映自己的。名叫彼得的人把自己当作人,只是由于他把名叫保罗的人看作是和自己相同的。因此,对彼得说来,这整个保罗以他保罗的肉体成为人这个物种的表现形式。}商品A同作为价值体,作为人类劳动的化身的商品B发生关系,就使B的使用价值成为表现A自己价值的材料。在商品B的使用价值上这样表现出来的商品B的价值,具有相对价值形式。
                
                \paragraph{相对价值形式的量的规定性}

                凡是价值要被表现的商品,都是一定量的使用物品,如15舍费耳小麦、100磅咖啡等等。这一定量的商品包含着一定量的人类劳动。因而,价值形式不只是要表现价值,而且要表现一定量的价值,即价值量。因此,在商品A和商品B如麻布和上衣的价值关系中,上衣这种商品不仅作为一般价值体被看作在质上同麻布相等,而且是作为一定量的价值体或等价物如1件上衣被看作同一定量的麻布如20码麻布相等。
                
                “20码麻布=1件上衣,或20码麻布值1件上衣”这一等式的前提是:1件上衣和20码麻布正好包含有同样多的价值实体。就是说,这两个商品量耗费了同样多的劳动或等量的劳动时间。但是生产20码麻布或1件上衣的必要劳动时间,是随着织或缝的生产力的变化而变化的。现在我们要较详细地研究一下这种变化对价值量的相对表现的影响。
                
                \uppercase\expandafter{\romannumeral1}.麻布的价值起了变化\footnote{“价值”一词在这里是用来指一定量的价值即价值量,前面有的地方已经这样用过。},上衣的价值不变。如果生产麻布的必要劳动时间由于种植亚麻的土地肥力下降而增加一倍,那末麻布的价值也就增大一倍。这时不是20码麻布=1件上衣,而是20码麻布=2件上衣,因为现在1件上衣包含的劳动时间只有20码麻布的一半。相反地,如果生产麻布的必要劳动时间由于织机改良而减少一半,那末,麻布的价值也就减低一半。这样,现在是20码麻布=1/2件上衣。可见,在商品B的价值不变时,商品A的相对价值即它表现在商品B上的价值的增减,与商品A的价值成正比。
                
                \uppercase\expandafter{\romannumeral2}.麻布的价值不变,上衣的价值起了变化。在这种情况下,如果生产上衣的必要劳动时间由于羊毛歉收而增加一倍,现在不是20码麻布=1件上衣,而是20码麻布=1/2件上衣。相反地,如果上衣的价值减少一半,那末,20码麻布=2件上衣。因此,在商品A的价值不变时,它的相对的、表现在商品B上的价值的增减,与商品B的价值变化成反比。
                
                我们把\uppercase\expandafter{\romannumeral1}、\uppercase\expandafter{\romannumeral2}类的各种情形对照一下就会发现,相对价值的同样的量的变化可以由完全相反的原因造成。所以,20码麻布=1件上衣变为:1.20码麻布=2件上衣,或者是由于麻布的价值增加一倍,或者是由于上衣的价值减低一半;2.20码麻布=1/2件上衣,或者是由于麻布的价值减低一半,或者是由于上衣的价值增加一倍。
                
                \uppercase\expandafter{\romannumeral3}.生产麻布和上衣的必要劳动量可以按照同一方向和同一比例同时发生变化。在这种情况下,不管这两种商品的价值发生什么变动,依旧是20码麻布=1件上衣。只有把它们同价值不变的第三种商品比较,才会发现它们的价值的变化。如果所有商品的价值都按同一比例同时增减,它们的相对价值就保持不变。它们的实际的价值变化可以由以下这个事实看出:在同样的劳动时间内,现在提供的商品量都比过去多些或少些。
                
                \uppercase\expandafter{\romannumeral4}.生产麻布和上衣的各自的必要劳动时间,从而它们的价值,可以按照同一方向但以不同的程度同时发生变化,或者按照相反的方向发生变化,等等。这种种可能的组合对一种商品的相对价值的影响,根据\uppercase\expandafter{\romannumeral1}、\uppercase\expandafter{\romannumeral2}、\uppercase\expandafter{\romannumeral3}类的情况就可以推知。
                
                可见,价值量的实际变化不能明确地,也不能完全地反映在价值量的相对表现即相对价值量上。即使商品的价值不变,它的相对价值也可能发生变化。即使商品的价值发生变化,它的相对价值也可能不变,最后,商品的价值量和这个价值量的相对表现同时发生的变化,完全不需要一致。\footnote{第2版注:庸俗经济学以惯有的机警利用了价值量和它的相对表现之间的这种不一致现象。例如:“如果承认,A由于同它相交换的B提高而降低,虽然这时在A上所耗费的劳动并不比以前少,这样,你们的一般价值原理就破产了……如果承认,由于与B相对而言,A的价值提高,所以与A相对而言,B的价值就降低,那末,李嘉图提出的关于商品的价值总是取决于商品所体现的劳动量这个大原理就站不住脚了;因为既然A的费用的变化不仅改变了本身的价值(与同它相交换的B相对而言),而且也改变了B的价值(与A的价值相对而言),虽然生产B所需要的劳动量并未发生任何变化,那末,不仅确认商品生产所耗费的劳动量调节商品价值的学说要破产,而且断言商品的生产费用调节商品价值的学说也要破产。”(约·布罗德赫斯特《政治经济学》1842年伦敦版第11、14页)
                
                布罗德赫斯特先生也可以说:看看10/20、10/50、10/100等等分数罢。即使10这个数字不变,但它的相对量,它与分母20、50、100相对而言的量却不断下降。可见,整数(例如10)的大小由它包含的单位数来“调节”这个大原理破产了。}

            \subsubsection{等价形式}

            我们说过,当商品A(麻布)通过不同种商品B(上衣)的使用价值表现自己的价值时,它就使商品B取得一种特殊的价值形式,即等价形式。商品麻布显示出它自身的价值,是通过上衣没有取得与自己的物体形式不同的价值形式而与它相等。这样,麻布表现出它自身具有价值,实际上是通过上衣能与它直接交换。因此,一个商品的等价形式就是它能与另一个商品直接交换的形式。
            
            如果一种商品例如上衣成了另一种商品例如麻布的等价物,上衣因而获得了一种特殊的属性,即处于能够与麻布直接交换的形式,那末,这根本没有表明上衣与麻布交换的比例。既然麻布的价值量已定,这个比例就取决于上衣的价值量。不管是上衣表现为等价物,麻布表现为相对价值,还是相反,麻布表现为等价物,上衣表现为相对价值,上衣的价值量总是取决于生产它的必要劳动时间,因而和它的价值形式无关。但是一当上衣这种商品在价值表现中取得等价物的地位,它的价值量就不是作为价值量来表现了。在价值等式中,上衣的价值量不如说只是当作某物的一定的量。
            
            例如,40码麻布“值”什么呢?2件上衣。因为上衣这种商品在这里起着等价物的作用,作为使用价值的上衣与麻布相对立时是充当价值体,所以,一定量的上衣也就足以表现麻布的一定的价值量。因此,两件上衣能够表现40码麻布的价值量,但是两件上衣决不能表现它们自己的价值量,即上衣的价值量。在价值等式中,等价物始终只具有某物即某种使用价值的单纯的量的形式,对这一事实的肤浅了解,使贝利同他的许多先驱者和后继者都误认为价值表现只是一种量的关系。其实,商品的等价形式不包含价值的量的规定。
            
            在考察等价形式时看见的第一个特点,就是使用价值成为它的对立面即价值的表现形式。
            
            商品的自然形式成为价值形式。但是请注意,对商品B(上衣、小麦或铁等等)来说,这种转换只有在任何别的商品A(麻布等等)与它发生价值关系时,只有在这种关系中才能实现。因为任何商品都不能把自己当作等价物来同自己发生关系,因而也不能用它自己的自然外形来表现它自己的价值,所以它必须把另一商品当作等价物来同它发生关系,或者使另一商品的自然外形成为它自己的价值形式。
            
            为了说明这一点,可以用衡量商品体本身即使用价值的尺度作例子。塔糖是物体,所以是重的,因而有重量,但是我们看不见也摸不着塔糖的重量。现在我们拿一些不同的铁块来,这些铁块的重量是预先确定了的。铁的物体形式,就其自身来说,同塔糖的物体形式一样,不是重的表现形式。要表现塔糖是重的,我们就要使它和铁发生重量关系。在这种关系中,铁充当一种只表示重而不表示别的东西的物体。因此,铁的量充当糖的重量尺度,对糖这个物体来说,它只是重的体现,重的表现形式。铁只是在糖或其他任何要测定重量的物体同它发生重量关系的时候,才起这种作用。如果两种物都没有重,它们就不能发生这种关系,因此一种物就不能成为另一种物的重的表现。如果把二者放在天平上,我们就会在实际上看到,当作有重的物,它们是相同的,因而在一定的比例上也具有同样的重量。铁这个物体作为重量尺度,对于塔糖来说,只代表重,同样,在我们的价值表现中,上衣这个物体对于麻布来说,也只代表价值。
            
            但是,类比只能到此为止。在塔糖的重量表现中,铁代表两个物体共有的自然属性,即它们的重,而在麻布的价值表现中,上衣代表这两种物的超自然属性,即它们的价值,某种纯粹社会的东西。
            
            一种商品例如麻布的相对价值形式,把自己的价值表现为一种与自己的物体和物体属性完全不同的东西,例如表现为与上衣相同的东西,因此,这个表现本身就说明其中隐藏着某种社会关系。等价形式却相反。等价形式恰恰在于:商品体例如上衣这个物本身就表现价值,因而天然就具有价值形式。当然,只是在商品麻布把商品上衣当作等价物的价值关系中,才是这样。\footnote{这种反思的规定是十分奇特的。例如,这个人所以是国王,只因为其他人作为臣民同他发生关系。反过来,他们所以认为自己是臣民,是因为他是国王。}但是,既然一物的属性不是由该物同他物的关系产生,而只是在这种关系中表现出来,因此上衣似乎天然具有等价形式,天然具有能与其他商品直接交换的属性,就象它天然具有重的属性或保暖的属性一样。从这里就产生了等价形式的谜的性质,这种性质只是在等价形式以货币这种完成的形态出现在政治经济学家的面前的时候,才为他的资产阶级的短浅的眼光所注意。这时他用不太耀眼的商品代替金银,并以一再满足的心情反复列举各种曾经充当过商品等价物的普通商品,企图以此来说明金银的神秘性质。他没有料到,最简单的价值表现,如20码麻布=1件上衣,就已经提出了等价形式的谜让人们去解决。
            
            充当等价物的商品的物体总是当作抽象人类劳动的化身,同时又总是某种有用的、具体的劳动的产品。因此,这种具体劳动就成为抽象人类劳动的表现。例如,如果上衣只当作抽象人类劳动的实现,那末,在上衣内实际地实现的缝劳动就只当作抽象人类劳动的实现形式。在麻布的价值表现中,缝劳动的有用性不在于造了衣服,从而造了人\footnote{原文套用了德国谚语《Kleider machen Leute》,直译是:“衣服造人”,转义是:人靠衣装。——译者注},而在于造了一种物体,使人们能看出它是价值,因而是与物化在麻布价值内的劳动毫无区别的那种劳动的凝结。要造这样一面反映价值的镜子,缝劳动本身就必须只是反映它作为人类劳动的这种抽象属性。
            
            缝的形式同织的形式一样,都是人类劳动力的耗费。因此,二者都具有人类劳动的一般属性,因而在一定的情况下,比如在价值的生产上,就可以只从这个角度来考察。这并不神秘。但是在商品的价值表现上事情却反过来了。例如,为了表明织不是在它作为织这个具体形式上,而是在它作为人类劳动这个一般属性上形成麻布的价值,我们就要把缝这种制造麻布的等价物的具体劳动,作为抽象人类劳动的可以捉摸的实现形式与织相对立。
            
            可见,等价形式的第二个特点,就是具体劳动成为它的对立面即抽象人类劳动的表现形式。
            
            既然这种具体劳动,即缝,只是当作无差别的人类劳动的表现,它也就具有与别种劳动即麻布中包含的劳动等同的形式,因而,尽管它同其他一切生产商品的劳动一样是私人劳动,但终究是直接社会形式上的劳动。正因为这样,它才表现在一种能与别种商品直接交换的产品上。可见,等价形式的第三个特点,就是私人劳动成为它的对立面的形式,成为直接社会形式的劳动。
            
            如果我们回顾一下一位伟大的研究家,等价形式的后两个特点就会更容易了解。这位研究家最早分析了许多思维形式、社会形式和自然形式,也最早分析了价值形式。他就是亚里士多德。
            
            首先,亚里士多德清楚地指出,商品的货币形式不过是简单价值形式——一种商品的价值通过任何别一种商品来表现——的进一步发展的形态,因为他说:
            
            \begin{center}
                “5张床=1间屋” \\
                “无异于”:\\
                “5张床=若干货币”。
            \end{center}
            
            
            其次,他看到:包含着这个价值表现的价值关系本身,要求屋必须在质上与床等同,这两种感觉上不同的物,如果没有这种本质上的等同性,就不能作为可通约的量而互相发生关系。他说:“没有等同性,就不能交换,没有可通约性,就不能等同。”但是他到此就停下来了,没有对价值形式作进一步分析。“实际上,这样不同种的物是不能通约的”,就是说,它们不可能在质上等同。这种等同只能是某种和物的真实性质相异的东西,因而只能是“应付实际需要的手段”[34]。
            
            可见,亚里士多德自己告诉了我们,是什么东西阻碍他作进一步的分析,这就是缺乏价值概念。这种等同的东西,也就是屋在床的价值表现中对床来说所代表的共同的实体是什么呢?亚里士多德说,这种东西“实际上是不可能存在的”。为什么呢?只要屋代表床和屋二者中真正等同的东西,对床来说屋就代表一种等同的东西。这就是人类劳动。
            
            但是,亚里士多德不能从价值形式本身看出,在商品价值形式中,一切劳动都表现为等同的人类劳动,因而是同等意义的劳动,这是因为希腊社会是建立在奴隶劳动的基础上的,因而是以人们之间以及他们的劳动力之间的不平等为自然基础的。价值表现的秘密,即一切劳动由于而且只是由于都是一般人类劳动而具有的等同性和同等意义,只有在人类平等概念已经成为国民的牢固的成见的时候,才能揭示出来。而这只有在这样的社会里才有可能,在那里,商品形式成为劳动产品的一般形式,从而人们彼此作为商品所有者的关系成为占统治地位的社会关系。亚里士多德在商品的价值表现中发现了等同关系,正是在这里闪耀出他的天才的光辉。只是他所处的社会的历史限制,使他不能发现这种等同关系“实际上”是什么。

            
            \subsubsection{简单价值形式的总体}

            一个商品的简单价值形式包含在它与一个不同种商品的价值关系或交换关系中。商品A的价值,通过商品B能与商品A直接交换而在质上得到表现,通过一定量的商品B能与既定量的商品A交换而在量上得到表现。换句话说,一个商品的价值是通过它表现为“交换价值”而得到独立的表现。在本章的开头,我们曾经依照通常的说法,说商品是使用价值和交换价值,严格说来,这是不对的。商品是使用价值或使用物品和“价值”。一个商品,只要它的价值取得一个特别的、不同于它的自然形式的表现形式,即交换价值形式,它就表现为这样的二重物。孤立地考察,它绝没有这种形式,而只有同第二个不同种的商品发生价值关系或交换关系时,它才具有这种形式。只要我们知道了这一点,上述说法就没有害处,而只有简便的好处。
            
            我们的分析表明,商品的价值形式或价值表现由商品价值的本性产生,而不是相反,价值和价值量由它们的作为交换价值的表现方式产生。但是,这正是重商主义者和他们的现代复兴者费里埃、加尼耳之流\footnote{第2版注:弗·路·奥·费里埃(海关副督查)《论政府和贸易的相互关系》1805年巴黎版。沙尔·加尼耳《论政治经济学的各种体系》1821年巴黎第2版。}的错觉,也是他们的反对者现代自由贸易贩子巴师夏之流的错觉。重商主义者看重价值表现的质的方面,也就是看重在货币上取得完成形态的商品等价形式,相反地,必须以任何价格出售自己的商品的现代自由贸易贩子,则看重相对价值形式的量的方面。因此,在他们看来,商品的价值和价值量只存在于由交换关系引起的表现中,也就是只存在于每日行情表中。苏格兰人麦克劳德,由于他的职责是用尽可能博学的外衣来粉饰伦巴特街[35]的杂乱的观念,而成了迷信的重商主义者和开明的自由贸易贩子之间的一个成功的综合。
            
            更仔细地考察一下商品A同商品B的价值关系中所包含的商品A的价值表现,就会知道,在这一关系中商品A的自然形式只是充当使用价值的形态,而商品B的自然形式只是充当价值形式或价值形态。这样,潜藏在商品中的使用价值和价值的内部对立,就通过外部对立,即通过两个商品的关系表现出来了,在这个关系中,价值要被表现的商品只是直接当作使用价值,而另一个表现价值的商品只是直接当作交换价值。所以,一个商品的简单的价值形式,就是该商品中所包含的使用价值和价值的对立的简单表现形式。
            
            在一切社会状态下,劳动产品都是使用物品,但只是历史上一定的发展时代,也就是使生产一个使用物所耗费的劳动表现为该物的“对象的”属性即它的价值的时代,才使劳动产品转化为商品。由此可见,商品的简单价值形式同时又是劳动产品的简单商品形式,因此,商品形式的发展是同价值形式的发展一致的。
            
            一看就知道,简单价值形式是不充分的,是一种胚胎形式,它只有通过一系列的形态变化,才成熟为价格形式。
            
            商品A的价值表现在某种商品B上,只是使商品A的价值同它自己的使用价值区别开来,因此也只是使商品A同某一种与它自身不同的商品发生交换关系,而不是表现商品A同其他一切商品的质的等同和量的比例。与一个商品的简单相对价值形式相适应的,是另一个商品的个别等价形式。所以,在麻布的相对价值表现中,上衣只是对麻布这一种商品来说,具有等价形式或能直接交换的形式。
            
            然而个别的价值形式会自行过渡到更完全的形式。通过个别的价值形式,商品A的价值固然只是表现在一个别种商品上,但是这后一个商品不论是哪一种,是上衣、铁或小麦等等,都完全一样。随着同一商品和这种或那种不同的商品发生价值关系,也就产生它的种种不同的简单价值表现。\footnote{第2版注:例如在荷马的著作中,一物的价值是通过一系列各种不同的物来表现的。}它可能有的价值表现的数目,只受与它不同的商品种类的数目的限制。这样,商品的个别的价值表现就转化为一个可以不断延长的、不同的简单价值表现的系列。

        \subsection{总和的或扩大的价值形式}

        \begin{center}
            z量商品A=u量商品B,或=v量商品C,或=w量商品D,或=x量商品E,或=其他\\
            (20码麻布=1件上衣,或=10磅茶叶,或=40磅咖啡,或=1夸特小麦,或=2盎斯金,或=1/2吨铁,或=其他)    
        \end{center}
        
            \subsubsection{扩大的相对价值形式}

            现在,一种商品例如麻布的价值表现在商品世界的其他无数的元素上。每一种其他的商品体都成为反映麻布价值的镜子。\footnote{因此,如果麻布的价值用上衣来表现,我们就说麻布的上衣价值。如果麻布的价值用谷物来表现,我们就说麻布的谷物价值,依此类推。每一个这种表现都意味着,在上衣、谷物等等的使用价值上表现出来的是麻布的价值。“因为每种商品的价值都表示该商品在交换中的关系,所以根据它用来比较的商品,我们可以称它的价值为……谷物价值、呢绒价值;因此,有千万种价值,有多少种商品,就有多少种价值,它们都同样是现实的,又都同样是名义的。”(《对价值的本质、尺度和原因的批判研究,主要是论李嘉图先生及其信徒的著作》,《略论意见的形成和发表》一书的作者著,1825年伦敦版第39页)这部在英国曾经轰动一时的匿名著作的作者赛·贝利以为,只要这样指出同一商品价值具有种种不同的相对表现,就消除了规定价值概念的任何可能。虽然他十分浅薄,但却触及了李嘉图学说的弱点,李嘉图学派例如在《韦斯明斯特评论》上攻击贝利时流露的愤激情绪,就证明了这一点。}这样,这个价值本身才真正表现为无差别的人类劳动的凝结。因为形成这个价值的劳动现在十分清楚地表现为这样一种劳动,其他任何一种人类劳动都与之等同,而不管其他任何一种劳动具有怎样的自然形式,即不管它是物化在上衣、小麦、铁或金等等之中。因此,现在麻布通过自己的价值形式,不再是只同另一种商品发生社会关系,而是同整个商品世界发生社会关系。作为商品,它是这个世界的一个公民。同时,商品价值表现的无限的系列表明,商品价值是同它借以表现的使用价值的特殊形式没有关系的。
            
            在第一种形式即20码麻布=1件上衣中,这两种商品能以一定的量的比例相交换,可能是偶然的事情。相反地,在第二种形式中,一个根本不同于偶然现象并且决定着这种偶然现象的背景马上就显露出来了。麻布的价值无论是表现在上衣、咖啡或铁等等无数千差万别的、属于各个不同所有者的商品上,总是一样大的。两个单个商品所有者之间的偶然关系消失了。显然,不是交换调节商品的价值量,恰好相反,是商品的价值量调节商品的交换比例。

            \subsubsection{特殊等价形式}

            每一种商品,上衣、茶叶、小麦、铁等等,都在麻布的价值表现中充当等价物,因而充当价值体。每一种这样的商品的一定的自然形式,现在都成为一个特殊的等价形式,与其他许多特殊等价形式并列。同样,种种不同的商品体中所包含的多种多样的一定的、具体的、有用的劳动,现在只是一般人类劳动的同样多种的特殊的实现形式或表现形式。

            \subsubsection{总和的或扩大的价值形式的缺点}

            第一,商品的相对价值表现是未完成的,因为它的表现系列永无止境。每当新出现一种商品,从而提供一种新的价值表现的材料时,由一个个的价值等式连结成的锁链就会延长。第二,这条锁链形成一幅由互不关联的而且种类不同的价值表现拼成的五光十色的镶嵌画。最后,象必然会发生的情形一样,如果每一种商品的相对价值都表现在这个扩大的形式中,那末,每一种商品的相对价值形式都是一个不同于任何别的商品的相对价值形式的无穷无尽的价值表现系列。——扩大的相对价值形式的缺点反映在与它相适应的等价形式中。既然每一种商品的自然形式在这里都是一个特殊的等价形式,与无数别的特殊等价形式并列,所以只存在着有局限性的等价形式,其中每一个都排斥另一个。同样,每个特殊的商品等价物中包含的一定的、具体的、有用的劳动,都只是人类劳动的特殊的因而是不充分的表现形式。诚然,人类劳动在这些特殊表现形式的总和中,获得自己的完全的或者总和的表现形式。但是它还没有获得统一的表现形式。
            
            扩大的相对价值形式只是由简单的相对价值表现的总和,或第一种形式的等式的总和构成,例如:
            
            \begin{center}
                20码麻布=1件上衣,\\
                20码麻布=10磅茶叶,等等。
            \end{center}
            
            
            但是每一个这样的等式倒转过来也包含着一个同一的等式:
            
            \begin{center}
                1件上衣=20码麻布,\\
                10磅茶叶=20码麻布,等等。
            \end{center}
            
            事实上,如果一个人用他的麻布同其他许多商品交换,从而把麻布的价值表现在一系列其他的商品上,那末,其他许多商品所有者也就必然要用他们的商品同麻布交换,从而把他们的各种不同的商品的价值表现在同一个第三种商品麻布上。——因此,把20码麻布=1件上衣,或=10磅茶叶,或=其他等等这个系列倒转过来,也就是说,把事实上已经包含在这个系列中的相反关系表示出来,我们就得到:

        \subsection{一般价值形式}

        \begin{center}
            \begin{tabular}{rrr}
                1件上衣&\multirow{8}*{=}&\multirow{8}*{20码麻布}\\
                10磅茶叶& &\\
                40磅咖啡& &\\
                1夸特小麦& &\\
                2盎斯金& &\\
                $\frac{1}{2}$吨铁& &\\
                x量商品A& &\\
                其他商品& &\\
            \end{tabular}
        \end{center} 

            \subsubsection{价值形式的变化了的性质}

            现在,商品价值的表现:1.是简单的,因为都是表现在唯一的商品上;2.是统一的,因为都是表现在同一的商品上。它们的价值形式是简单的和共同的,因而是一般的。
            
            第一种形式和第二种形式二者都只是使一种商品的价值表现为一种与它自身的使用价值或商品体不同的东西。
            
            第一种形式提供的价值等式是:1件上衣=20码麻布,10磅茶叶=1/2吨铁,等等。上衣的价值表现为与麻布等同,茶叶的价值表现为与铁等同,等等,但是与麻布等同和与铁等同——上衣和茶叶各自的这种价值表现是不相同的,正如麻布和铁不相同一样。很明显,这种形式实际上只是在最初交换阶段,也就是在劳动产品通过偶然的、间或的交换而转化为商品的阶段才出现。
            
            第二种形式比第一种形式更完全地把一种商品的价值同它自身的使用价值区别开来,因为例如上衣的价值现在是在一切可能的形式上与它的自然形式相对立,上衣的价值现在与麻布等同,与铁等同,与茶叶等同,与其他一切东西等同,只是不与上衣等同。另一方面,在这里商品的任何共同的价值表现都直接被排除了,因为在每一种商品的价值表现中,其他一切商品现在都只是以等价物的形式出现。扩大的价值形式,事实上是在某种劳动产品例如牲畜不再是偶然地而已经是经常地同其他不同的商品交换的时候,才出现的。
            
            新获得的形式使商品世界的价值表现在从商品世界中分离出来的同一种商品上,例如表现在麻布上,因而使一切商品的价值都通过它们与麻布等同而表现出来。每个商品的价值作为与麻布等同的东西,现在不仅与它自身的使用价值相区别,而且与一切使用价值相区别,正因为这样才表现为它和一切商品共有的东西。因此,只有这种形式才真正使商品作为价值互相发生关系,或者使它们互相表现为交换价值。
            
            前两种形式表现一种商品的价值,或者是通过一个不同种的商品,或者是通过许多种与它不同的商品构成的系列。在这两种情况下,使自己取得一个价值形式可以说是个别商品的私事,它完成这件事情是不用其他商品帮助的。对它来说,其他商品只是起着被动的等价物的作用。相反地,一般价值形式的出现只是商品世界共同活动的结果。一种商品所以获得一般的价值表现,只是因为其他一切商品同时也用同一个等价物来表现自己的价值,而每一种新出现的商品都要这样做。这就表明,由于商品的价值对象性只是这些物的“社会存在”,所以这种对象性也就只能通过它们全面的社会关系来表现,因而它们的价值形式必须是社会公认的形式。
            
            现在,一切商品,在与麻布等同的形式上,不仅表现为在质上等同,表现为价值,而且同时也表现为在量上可以比较的价值量。由于它们都通过同一个材料,通过麻布来反映自己的价值量,这些价值量也就互相反映。例如,10磅茶叶=20码麻布,40磅咖啡=20码麻布。因此,10磅茶叶=40磅咖啡。或者说,一磅咖啡所包含的价值实体即劳动,只等于一磅茶叶所包含的1/4。
            
            商品世界的一般的相对价值形式,使被排挤出商品世界的等价物商品即麻布,获得了一般等价物的性质。麻布自身的自然形式是这个世界的共同的价值形态,因此,麻布能够与其他一切商品直接交换。它的物体形式是当作一切人类劳动的可以看得见的化身,一般的社会的蛹化。同时,织,这种生产麻布的私人劳动,也就处于一般社会形式,处于与其他一切劳动等同的形式。构成一般价值形式的无数等式,使实现在麻布中的劳动,依次等于包含在其他商品中的每一种劳动,从而使织成为一般人类劳动的一般表现形式。这样,物化在商品价值中的劳动,不仅消极地表现为被抽去了实在劳动的一切具体形式和有用属性的劳动。它本身的积极的性质也清楚地表现出来了。这就是把一切实在劳动化为它们共有的人类劳动的性质,化为人类劳动力的耗费。
            
            把劳动产品表现为只是无差别人类劳动的凝结物的一般价值形式,通过自身的结构表明,它是商品世界的社会表现。因此,它清楚地告诉我们,在这个世界中,劳动的一般的人类的性质形成劳动的特殊的社会的性质。

            \subsubsection{相对价值形式和等价形式的发展关系}

            等价形式的发展程度是同相对价值形式的发展程度相适应的。但是必须指出,等价形式的发展只是相对价值形式发展的表现和结果。
            
            一种商品的简单的或个别的相对价值形式使另一种商品成为个别的等价物。扩大的相对价值形式,即一种商品的价值在其他一切商品上的表现,赋予其他一切商品以种种不同的特殊等价物的形式。最后,一种特殊的商品获得一般等价形式,是因为其他一切商品使它成为它们统一的、一般的价值形式的材料。
            
            价值形式发展到什么程度,它的两极即相对价值形式和等价形式之间的对立,也就发展到什么程度。
            
            第一种形式——20码麻布=1件上衣——就已经包含着这种对立,但没有使这种对立固定下来。我们从等式的左边读起,麻布是相对价值形式,上衣是等价形式,从等式的右边读起,上衣是相对价值形式,麻布是等价形式。在这里,要把握住两极的对立还相当困难。
            
            在第二种形式中,每一次总是只有一种商品可以完全展开它的相对价值,或者说,它自身具有扩大的相对价值形式,是因为而且只是因为其他一切商品与它相对立,处于等价形式。在这里,不能再变换价值等式(例如20码麻布=1件上衣,或=10磅茶叶,或=1夸特小麦等等)的两边的位置,除非改变价值等式的全部性质,使它从总和的价值形式变成一般的价值形式。
            
            最后,后面一种形式,即第三种形式,给商品世界提供了一般的社会的相对价值形式,是因为而且只是因为除了一个唯一的例外,商品世界的一切商品都不能具有一般等价形式。因此,一种商品如麻布处于能与其他一切商品直接交换的形式,或者说,处于直接的社会的形式,是因为而且只是因为其他一切商品都不是处于这种形式。\footnote{实际上从一般的能直接交换的形式决不可能看出,它是一种对立的商品形式,是同不能直接交换的形式分不开的,就象一块磁铁的阳极同阴极分不开一样。因此,设想能够同时在一切商品上打上能直接交换的印记,就象设想能够把一切天主教徒都变成教皇一样。对于把商品生产看作人类自由和个人独立的顶峰的小资产者来说,去掉与这种形式相联系的缺点,特别是去掉商品的不能直接交换的性质,那当然是再好不过的事。蒲鲁东的社会主义就是对这种庸俗空想的描绘;我在别的地方曾经指出[36],这种社会主义连首创的功绩也没有,在它以前很久,就由格雷、布雷以及其他人更好地阐述过了。在今天,这并不妨碍这种智慧以“科学”的名义在一定范围内蔓延开来。没有一个学派比蒲鲁东学派更会滥用“科学”这个字眼了,因为
            \begin{center}
                “缺乏概念的地方
            
                字眼就及时出现”[37]。
            \end{center}
            }
            
            相反地,充当一般等价物的商品则不能具有商品世界的统一的、从而是一般的相对价值形式。如果麻布,或任何一种处于一般等价形式的商品,要同时具有一般的相对价值形式,那末,它必须自己给自己充当等价物。于是我们得到的就是20码麻布=20码麻布,这是一个既不表现价值也不表现价值量的同义反复。要表现一般等价物的相对价值,我们就必须把第三种形式倒过来。一般等价物没有与其他商品共同的相对价值形式,它的价值相对地表现在其他一切商品体的无限的系列上。因此,扩大的相对价值形式,即第二种形式,现在表现为等价物商品特有的相对价值形式。

            \subsubsection{从一般价值形式到货币形式的过渡}

            一般等价形式是价值的一种形式。因此,它可以属于任何一种商品。另一方面,一种商品处于一般等价形式(第三种形式),是因为而且只是因为它被其他一切商品当作等价物排挤出来。这种排挤最终限制在一种特殊的商品上,从这个时候起,商品世界的统一的相对价值形式才获得客观的固定性和一般的社会效力。
            
            等价形式同这种特殊商品的自然形式社会地结合在一起,这种特殊商品成了货币商品,或者执行货币的职能。在商品世界起一般等价物的作用就成了它特有的社会职能,从而成了它的社会独占权。在第二种形式中充当麻布的特殊等价物,而在第三种形式中把自己的相对价值共同用麻布来表现的各种商品中间,有一种商品在历史过程中夺得了这个特权地位,这就是金。因此,我们在第三种形式中用商品金代替商品麻布,就得到:

        \subsection{货币形式}

        \begin{center}
            \begin{tabular}{rrr}
                20码麻布&\multirow{7}*{=}&\multirow{7}*{2盎斯金}\\
                1件上衣& &\\
                10磅茶叶& &\\
                40磅咖啡& &\\
                1夸特小麦& &\\
                $\frac{1}{2}$吨铁& &\\
                x量商品A& &\\
            \end{tabular}
        \end{center} 

        在第一种形式过渡到第二种形式,第二种形式过渡到第三种形式的时候,都发生了本质的变化。而第四种形式与第三种形式的唯一区别,只是金代替麻布取得了一般等价形式。金在第四种形式中同麻布在第三种形式中一样,都是一般等价物。唯一的进步是在于:能直接地一般地交换的形式,即一般等价形式,现在由于社会的习惯最终地同商品金的特殊的自然形式结合在一起了。
        
        金能够作为货币与其他商品相对立,只是因为它早就作为商品与它们相对立。与其他一切商品一样,它过去就起等价物的作用:或者是在个别的交换行为中起个别等价物的作用,或者是与其他商品等价物并列起特殊等价物的作用。渐渐地,它就在或大或小的范围内起一般等价物的作用。一当它在商品世界的价值表现中独占了这个地位,它就成为货币商品。只是从它已经成为货币商品的时候起,第四种形式才同第三种形式区别开来,或者说,一般价值形式才转化为货币形式。
        
        一种商品(如麻布)在已经执行货币商品职能的商品(如金)上的简单的相对的价值表现,就是价格形式。因此,麻布的“价格形式”是:
        \centerline{20码麻布=2盎斯金,}
        如果2盎斯金的铸币名称是2镑,那就是:

        \centerline{20码麻布=2镑。}
        
        理解货币形式的困难,无非是理解一般等价形式,从而理解一般价值形式即第三种形式的困难。第三种形式倒转过来,就化为第二种形式,即扩大的价值形式,而第二种形式的构成要素是第一种形式:20码麻布=1件上衣,或者x量商品A=y量商品B。因此,简单的商品形式是货币形式的胚胎。

    \section{商品的拜物教性质及其秘密}

    最初一看,商品好象是一种很简单很平凡的东西。对商品的分析表明,它却是一种很古怪的东西,充满形而上学的微妙和神学的怪诞。商品就它是使用价值来说,不论从它靠自己的属性来满足人的需要这个角度来考察,或者从它作为人类劳动的产品才具有这些属性这个角度来考察,都没有什么神秘的地方。很明显,人通过自己的活动按照对自己有用的方式来改变自然物质的形态。例如,用木头做桌子,木头的形状就改变了。可是桌子还是木头,还是一个普通的可以感觉的物。但是桌子一旦作为商品出现,就变成一个可感觉而又超感觉的物了。它不仅用它的脚站在地上,而且在对其他一切商品的关系上用头倒立着,从它的木脑袋里生出比它自动跳舞还奇怪得多的狂想。\footnote{我们想起了,当世界其他一切地方好象静止的时候,中国和桌子开始跳起舞来,以激励别人[38]。}
    
    可见,商品的神秘性质不是来源于商品的使用价值。同样,这种神秘性质也不是来源于价值规定的内容。因为,第一,不管有用劳动或生产活动怎样不同,它们都是人体的机能,而每一种这样的机能不管内容和形式如何,实质上都是人的脑、神经、肌肉、感官等等的耗费。这是一个生理学上的真理。第二,说到作为决定价值量的基础的东西,即这种耗费的持续时间或劳动量,那末,劳动的量可以十分明显地同劳动的质区别开来。在一切社会状态下,人们对生产生活资料所耗费的劳动时间必然是关心的,虽然在不同的发展阶段上关心的程度不同。\footnote{第2版注:在古日耳曼人中,一摩尔根土地的面积是按一天的劳动来计算的。因此,摩尔根又叫做Tagwerk〔一日的工作〕(或Tagwanne)(jurnale或jurnalis,terra jurnalis,jornalis或diurnalis),Mannwerk〔一人的工作〕,Mannskraft〔一人的力量〕,Mannsmaad,Mannshauet〔一人的收割量〕等等。见格奥尔格·路德维希·冯·毛勒《马尔克制度、农户制度、乡村制度和城市制度以及公共政权的历史概论》1854年慕尼黑版第129页及以下各页。}最后,一旦人们以某种方式彼此为对方劳动,他们的劳动也就取得社会的形式。可是,劳动产品一采取商品形式就具有的谜一般的性质究竟是从哪里来的呢?显然是从这种形式本身来的。人类劳动的等同性,取得了劳动产品的等同的价值对象性这种物的形式;用劳动的持续时间来计量的人类劳动力的耗费,取得了劳动产品的价值量的形式;最后,劳动的那些社会规定借以实现的生产者的关系,取得了劳动产品的社会关系的形式。
    
    可见,商品形式的奥秘不过在于:商品形式在人们面前把人们本身劳动的社会性质反映成劳动产品本身的物的性质,反映成这些物的天然的社会属性,从而把生产者同总劳动的社会关系反映成存在于生产者之外的物与物之间的社会关系。由于这种转换,劳动产品成了商品,成了可感觉而又超感觉的物或社会的物。正如一物在视神经中留下的光的印象,不是表现为视神经本身的主观兴奋,而是表现为眼睛外面的物的客观形式。但是在视觉活动中,光确实从一物射到另一物,即从外界对象射入眼睛。这是物理的物之间的物理关系。相反,商品形式和它借以得到表现的劳动产品的价值关系,是同劳动产品的物理性质以及由此产生的物的关系完全无关的。这只是人们自己的一定的社会关系,但它在人们面前采取了物与物的关系的虚幻形式。因此,要找一个比喻,我们就得逃到宗教世界的幻境中去。在那里,人脑的产物表现为赋有生命的、彼此发生关系并同人发生关系的独立存在的东西。在商品世界里,人手的产物也是这样。我把这叫做拜物教。劳动产品一旦作为商品来生产,就带上拜物教性质,因此拜物教是同商品生产分不开的。
    
    商品世界的这种拜物教性质,象以上分析已经表明的,是来源于生产商品的劳动所特有的社会性质。
    
    使用物品成为商品,只是因为它们是彼此独立进行的私人劳动的产品。这种私人劳动的总和形成社会总劳动。由于生产者只有通过交换他们的劳动产品才发生社会接触,因此,他们的私人劳动的特殊的社会性质也只有在这种交换中才表现出来。换句话说,私人劳动在事实上证实为社会总劳动的一部分,只是由于交换使劳动产品之间、从而使生产者之间发生了关系。因此,在生产者面前,他们的私人劳动的社会关系就表现为现在这个样子,就是说,不是表现为人们在自己劳动中的直接的社会关系,而是表现为人们之间的物的关系和物之间的社会关系。
    
    劳动产品只是在它们的交换中,才取得一种社会等同的价值对象性,这种对象性是与它们的感觉上各不相同的使用对象性相分离的。劳动产品分裂为有用物和价值物,实际上只是发生在交换已经十分广泛和十分重要的时候,那时有用物是为了交换而生产的,因而物的价值性质还在生产时就被注意到了。从那时起,生产者的私人劳动真正取得了二重的社会性质。一方面,生产者的私人劳动必须作为一定的有用劳动来满足一定的社会需要,从而证明它们是总劳动的一部分,是自然形成的社会分工体系的一部分。另一方面,只有在每一种特殊的有用的私人劳动可以同任何另一种有用的私人劳动相交换从而相等时,生产者的私人劳动才能满足生产者本人的多种需要。完全不同的劳动所以能够相等,只是因为它们的实际差别已被抽去,它们已被化成它们作为人类劳动力的耗费、作为抽象的人类劳动所具有的共同性质。私人生产者的头脑把他们的私人劳动的这种二重的社会性质,只是反映在从实际交易,产品交换中表现出来的那些形式中,也就是把他们的私人劳动的社会有用性,反映在劳动产品必须有用,而且是对别人有用的形式中;把不同种劳动的相等这种社会性质,反映在这些在物质上不同的物即劳动产品具有共同的价值性质的形式中。
    
    可见,人们使他们的劳动产品彼此当作价值发生关系,不是因为在他们看来这些物只是同种的人类劳动的物质外壳。恰恰相反,他们在交换中使他们的各种产品作为价值彼此相等,也就使他们的各种劳动作为人类劳动而彼此相等。他们没有意识到这一点,但是他们这样做了。\footnote{第2版注:因此,当加利阿尼说价值是人和人之间的一种关系时,他还应当补充一句:这是被物的外壳掩盖着的关系。(加利阿尼《货币论》,载于库斯托第编《意大利政治经济学名家文集》现代部分,1803年米兰版第3卷第221页)}价值没有在额上写明它是什么。不仅如此,价值还把每个劳动产品变成社会的象形文字。后来,人们竭力要猜出这种象形文字的涵义,要了解他们自己的社会产品的秘密,因为使用物品当作价值,正象语言一样,是人们的社会产物。后来科学发现,劳动产品作为价值,只是生产它们时所耗费的人类劳动的物的表现,这一发现在人类发展史上划了一个时代,但它决没有消除劳动的社会性质的物的外观。彼此独立的私人劳动的特殊的社会性质表现为它们作为人类劳动而彼此相等,并且采取劳动产品的价值性质的形式——商品生产这种特殊生产形式所独具的这种特点,在受商品生产关系束缚的人们看来,无论在上述发现以前或以后,都是永远不变的,正象空气形态在科学把空气分解为各种元素之后,仍然作为一种物理的物态继续存在一样。
    
    产品交换者实际关心的问题,首先是他用自己的产品能换取多少别人的产品,就是说,产品按什么样的比例交换。当这些比例由于习惯而逐渐达到一定的稳固性时,它们就好象是由劳动产品的本性产生的。例如,1吨铁和2盎斯金的价值相等,就象1磅金和1磅铁虽然有不同的物理属性和化学属性,但是重量相等一样。实际上,劳动产品的价值性质,只是通过劳动产品作为价值量发生作用才确定下来。价值量不以交换者的意志、设想和活动为转移而不断地变动着。在交换者看来,他们本身的社会运动具有物的运动形式。不是他们控制这一运动,而是他们受这一运动控制。要有十分发达的商品生产,才能从经验本身得出科学的认识,理解到彼此独立进行的、但作为自然形成的社会分工部分而互相全面依赖的私人劳动,不断地被化为它们的社会的比例尺度,这是因为在私人劳动产品的偶然的不断变动的交换关系中,生产这些产品的社会必要劳动时间作为起调节作用的自然规律强制地为自己开辟道路,就象房屋倒在人的头上时重力定律强制地为自己开辟道路一样。\footnote{“我们应该怎样理解这个只有通过周期性的革命才能为自己开辟道路的规律呢?这是一个以当事人的盲目活动为基础的自然规律。”(弗里德里希·恩格斯《政治经济学批评大纲》,载于阿尔诺德·卢格和卡尔·马克思编的《德法年鉴》1844年巴黎版[39])}因此,价值量由劳动时间决定是一个隐藏在商品相对价值的表面运动后面的秘密。这个秘密的发现,消除了劳动产品的价值量纯粹是偶然决定的这种假象,但是决没有消除这种决定所采取的物的形式。
    
    对人类生活形式的思索,从而对它的科学分析,总是采取同实际发展相反的道路。这种思索是从事后开始的,就是说,是从发展过程的完成的结果开始的。给劳动产品打上商品烙印、因而成为商品流通的前提的那些形式,在人们试图了解它们的内容而不是了解它们的历史性质(人们已经把这些形式看成是不变的了)以前,就已经取得了社会生活的自然形式的固定性。因此,只有商品价格的分析才导致价值量的决定,只有商品共同的货币表现才导致商品的价值性质的确定。但是,正是商品世界的这个完成的形式——货币形式,用物的形式掩盖了私人劳动的社会性质以及私人劳动者的社会关系,而不是把它们揭示出来。如果我说,上衣、皮靴等等把麻布当作抽象的人类劳动的一般化身而同它发生关系,这种说法的荒谬是一目了然的。但是当上衣、皮靴等等的生产者使这些商品同作为一般等价物的麻布(或者金银,这丝毫不改变问题的性质)发生关系时,他们的私人劳动同社会总劳动的关系正是通过这种荒谬形式呈现在他们面前。
    
    这种种形式恰好形成资产阶级经济学的各种范畴。对于这个历史上一定的社会生产方式即商品生产的生产关系来说,这些范畴是有社会效力的、因而是客观的思维形式。因此,一旦我们逃到其他的生产形式中去,商品世界的全部神秘性,在商品生产的基础上笼罩着劳动产品的一切魔法妖术,就立刻消失了。
    
    既然政治经济学喜欢鲁滨逊的故事\footnote{第2版注:甚至李嘉图也离不开他的鲁滨逊故事。“他让原始的渔夫和原始的猎人一下子就以商品所有者的身分,按照物化在鱼和野味的交换价值中的劳动时间的比例交换鱼和野味。在这里他犯了时代错误,他竟让原始的渔夫和猎人在计算他们的劳动工具时去查看1817年伦敦交易所通用的年息表。看来,除了资产阶级社会形式以外,‘欧文先生的平行四边形’[40]是他所知道的唯一的社会形式。”(卡尔·马克思《政治经济学批判》第38、39页[41])},那末就先来看看孤岛上的鲁滨逊吧。不管他生来怎样简朴,他终究要满足各种需要,因而要从事各种有用劳动,如做工具,制家具,养羊驼,捕鱼,打猎等等。关于祈祷一类事情我们在这里就不谈了,因为我们的鲁滨逊从中得到快乐,他把这类活动当作休息。尽管他的生产职能是不同的,但是他知道,这只是同一个鲁滨逊的不同的活动形式,因而只是人类劳动的不同方式。需要本身迫使他精确地分配自己执行各种职能的时间。在他的全部活动中,这种或那种职能所占比重的大小,取决于他为取得预期效果所要克服的困难的大小。经验告诉他这些,而我们这位从破船上抢救出表、账簿、墨水和笔的鲁滨逊,马上就作为一个道地的英国人开始记起账来。他的账本记载着他所有的各种使用物品,生产这些物品所必需的各种活动,最后还记载着他制造这种种一定量的产品平均耗费的劳动时间。鲁滨逊和构成他自己创造的财富的物之间的全部关系在这里是如此简单明了,甚至连麦·维尔特先生用不着费什么脑筋也能了解。但是,价值的一切本质上的规定都包含在这里了。
    
    现在,让我们离开鲁滨逊的明朗的孤岛,转到欧洲昏暗的中世纪去吧。在这里,我们看到的,不再是一个独立的人了,人都是互相依赖的:农奴和领主,陪臣和诸侯,俗人和牧师。物质生产的社会关系以及建立在这种生产的基础上的生活领域,都是以人身依附为特征的。但是正因为人身依附关系构成该社会的基础,劳动和产品也就用不着采取与它们的实际存在不同的虚幻形式。它们作为劳役和实物贡赋而进入社会机构之中。在这里,劳动的自然形式,劳动的特殊性是劳动的直接社会形式,而不是象在商品生产基础上那样,劳动的共性是劳动的直接社会形式。徭役劳动同生产商品的劳动一样,是用时间来计量的,但是每一个农奴都知道,他为主人服役而耗费的,是他本人的一定量的劳动力。缴纳给牧师的什一税,是比牧师的祝福更加清楚的。所以,无论我们怎样判断中世纪人们在相互关系中所扮演的角色,人们在劳动中的社会关系始终表现为他们本身之间的个人的关系,而没有披上物之间即劳动产品之间的社会关系的外衣。
    
    要考察共同的劳动即直接社会化的劳动,我们没有必要回溯到一切文明民族的历史初期都有过的这种劳动的原始的形式。\footnote{第2版注:“近来流传着一种可笑的偏见,认为原始的公社所有制是斯拉夫族特有的形式,甚至只是俄罗斯的形式。这种原始形式我们在罗马人、日耳曼人、克尔特人那里都可以见到,直到现在我们还能在印度人那里遇到这种形式的一整套图样,虽然其中一部分只留下残迹了。仔细研究一下亚细亚的、尤其是印度的公社所有制形式,就会得到证明,从原始的公社所有制的不同形式中,怎样产生出它的解体的各种形式。例如,罗马和日耳曼的私人所有制的各种原型,就可以从印度的公社所有制的各种形式中推出来。”(卡尔·马克思《政治经济学批判》第10页[42])}这里有个更近的例子,就是农民家庭为了自身的需要而生产粮食、牲畜、纱、麻布、衣服等等的那种农村家长制生产。对于这个家庭来说,这种种不同的物都是它的家庭劳动的不同产品,但它们不是互相作为商品发生关系。生产这些产品的种种不同的劳动,如耕、牧、纺、织、缝等等,在其自然形式上就是社会职能,因为这是这样一个家庭的职能,这个家庭就象商品生产一样,有它本身的自然形成的分工。家庭内的分工和家庭各个成员的劳动时间,是由性别年龄上的差异以及随季节而改变的劳动的自然条件来调节的。但是,用时间来计量的个人劳动力的耗费,在这里本来就表现为劳动本身的社会规定,因为个人劳动力本来就只是作为家庭共同劳动力的器官而发挥作用的。
    
    最后,让我们换一个方面,设想有一个自由人联合体,他们用公共的生产资料进行劳动,并且自觉地把他们许多个人劳动力当作一个社会劳动力来使用。在那里,鲁滨逊的劳动的一切规定又重演了,不过不是在个人身上,而是在社会范围内重演。鲁滨逊的一切产品只是他个人的产品,因而直接是他的使用物品。这个联合体的总产品是社会的产品。这些产品的一部分重新用作生产资料。这一部分依旧是社会的。而另一部分则作为生活资料由联合体成员消费。因此,这一部分要在他们之间进行分配。这种分配的方式会随着社会生产机体本身的特殊方式和随着生产者的相应的历史发展程度而改变。仅仅为了同商品生产进行对比,我们假定,每个生产者在生活资料中得到的份额是由他的劳动时间决定的。这样,劳动时间就会起双重作用。劳动时间的社会的有计划的分配,调节着各种劳动职能同各种需要的适当的比例。另一方面,劳动时间又是计量生产者个人在共同劳动中所占份额的尺度,因而也是计量生产者个人在共同产品的个人消费部分中所占份额的尺度。在那里,人们同他们的劳动和劳动产品的社会关系,无论在生产上还是在分配上,都是简单明了的。
    
    在商品生产者的社会里,一般的社会生产关系是这样的:生产者把他们的产品当作商品,从而当作价值来对待,而且通过这种物的形式,把他们的私人劳动当作等同的人类劳动来互相发生关系。对于这种社会来说,崇拜抽象人的基督教,特别是资产阶级发展阶段的基督教,如新教、自然神教等等,是最适当的宗教形式。在古亚细亚的、古希腊罗马的等等生产方式下,产品变为商品、从而人作为商品生产者而存在的现象,处于从属地位,但是共同体越是走向没落阶段,这种现象就越是重要。真正的商业民族只存在于古代世界的空隙中,就象伊壁鸠鲁的神只存在于世界的空隙中[43],或者犹太人只存在于波兰社会的缝隙中一样。这些古老的社会生产机体比资产阶级的社会生产机体简单明了得多,但它们或者以个人尚未成熟,尚未脱掉同其他人的自然血缘联系的脐带为基础,或者以直接的统治和服从的关系为基础。它们存在的条件是:劳动生产力处于低级发展阶段,与此相应,人们在物质生活生产过程内部的关系,即他们彼此之间以及他们同自然之间的关系是很狭隘的。这种实际的狭隘性,观念地反映在古代的自然宗教和民间宗教中。只有当实际日常生活的关系,在人们面前表现为人与人之间和人与自然之间极明白而合理的关系的时候,现实世界的宗教反映才会消失。只有当社会生活过程即物质生产过程的形态,作为自由结合的人的产物,处于人的有意识有计划的控制之下的时候,它才会把自己的神秘的纱幕揭掉。但是,这需要有一定的社会物质基础或一系列物质生存条件,而这些条件本身又是长期的、痛苦的历史发展的自然产物。
    
    诚然,政治经济学曾经分析了价值和价值量(虽然不充分\footnote{李嘉图对价值量的分析并不充分,——但已是最好的分析,——这一点人们将在本书第三卷和第四卷中看到。至于价值本身,古典政治经济学在任何地方也没有明确地和十分有意识地把体现为价值的劳动同体现为产品使用价值的劳动区分开。当然,古典政治经济学事实上是这样区分的,因为它有时从量的方面,有时从质的方面来考察劳动。但是,它从来没有意识到,劳动的纯粹量的差别是以它们的质的统一或等同为前提的,因而是以它们化为抽象人类劳动为前提的。例如,李嘉图就曾表示他同意德斯杜特·德·特拉西的说法。德斯杜特说:“很清楚,我们的体力和智力是我们唯一的原始的财富,因此,这些能力的运用,某种劳动,是我们的原始的财宝;凡是我们称为财富的东西,总是由这些能力的运用创造出来的……此外,这一切东西确实只代表创造它们的劳动,如果它们有价值,或者甚至有两种不同的价值,那也只能来源于创造它们的劳动的价值。”(李嘉图《政治经济学原理》1821年伦敦第3版第334页)我们只指出,李嘉图在德斯杜特的话中塞进了自己的更加深刻的思想。一方面,德斯杜特确实说过,凡是构成财富的东西都“代表创造它们的劳动”。但是另一方面,他又说,这一切东西的“两种不同的价值”(使用价值和交换价值)来自“劳动的价值”。这样,他就陷入庸俗经济学的平庸浅薄之中。庸俗经济学先假设一种商品(在这里是指劳动)的价值,然后再用这种价值去决定其他商品的价值。而李嘉图却把德斯杜特的话读作:劳动(而不是劳动的价值)既表现为使用价值,也表现为交换价值。不过他自己也不善于区别具有二重表现的劳动的二重性质,以致在关于《价值和财富,它们的不同性质》这整整一章中,不得不同让·巴·萨伊这个人的庸俗见解苦苦纠缠。因此,最后他不禁楞住了:在劳动是价值的源泉这一点上,德斯杜特虽然同他是一致的,可是另一方面,在价值概念上,德斯杜特却同萨伊是一致的。}),揭示了这些形式所掩盖的内容。但它甚至从来也没有提出过这样的问题:为什么这一内容要采取这种形式呢?为什么劳动表现为价值,用劳动时间计算的劳动量表现为劳动产品的价值量呢?\footnote{古典政治经济学的根本缺点之一,就是它始终不能从商品的分析,而特别是商品价值的分析中,发现那种正是使价值成为交换价值的价值形式。恰恰是古典政治经济学的最优秀的代表人物,象亚·斯密和李嘉图,把价值形式看成一种完全无关紧要的东西或在商品本性之外存在的东西。这不仅仅因为价值量的分析把他们的注意力完全吸引住了。还有更深刻的原因。劳动产品的价值形式是资产阶级生产方式的最抽象的、但也是最一般的形式,这就使资产阶级生产方式成为一种特殊的社会生产类型,因而同时具有历史的特征。因此,如果把资产阶级生产方式误认为是社会生产的永恒的自然形式,那就必然会忽略价值形式的特殊性,从而忽略商品形式及其进一步发展——货币形式、资本形式等等的特殊性。因此,我们发现,在那些完全同意用劳动时间来计算价值量的经济学家中间,对于货币即一般等价物的完成形态的看法是极为混乱和矛盾的。例如,在考察银行业时,这一点表现得特别明显,因为在这里关于货币的通常的定义已经不够用了。于是,与此相对立的,出现了复兴的重商主义体系(加尼耳等人),这一体系在价值中只看到社会形式,或者更确切地说,只看到这种社会形式的没有实体的外观。——在这里,我断然指出,我所说的古典政治经济学,是指从威·配第以来的一切这样的经济学,这种经济学与庸俗经济学相反,研究了资产阶级生产关系的内部联系。而庸俗经济学却只是在表面的联系内兜圈子,它为了对可以说是最粗浅的现象作出似是而非的解释,为了适应资产阶级的日常需要,一再反复咀嚼科学的经济学早就提供的材料。在其他方面,庸俗经济学则只限于把资产阶级生产当事人关于他们自己的最美好世界的陈腐而自负的看法加以系统化,赋以学究气味,并且宣布为永恒的真理。}一些公式本来在额上写着,它们是属于生产过程支配人而人还没有支配生产过程的那种社会形态的,但在政治经济学的资产阶级意识中,它们竟象生产劳动本身一样,成了不言而喻的自然必然性。因此,政治经济学对待资产阶级以前的社会生产机体形式,就象教父对待基督教以前的宗教一样。\footnote{“经济学家们在论断中采用的方式是非常奇怪的。他们认为只有两种制度:一种是人为的,一种是天然的。封建制度是人为的,资产阶级制度是天然的。在这方面,经济学家很象那些把宗教也分为两类的神学家。一切异教都是人们臆造的,而他们自己的教则是神的启示。——于是,以前是有历史的,现在再也没有历史了。”(卡尔·马克思《哲学的贫困。答蒲鲁东先生的〈贫困的哲学〉》1847年版第113页[44])巴师夏先生认为古代希腊人和罗马人专靠掠夺为生,这真是滑稽可笑。如果人们几百年来都靠掠夺为生,那就得经常有可供掠夺的东西,或者说,被掠夺的对象应当不断地被再生产出来。可见,希腊人和罗马人看来也要有某种生产过程,从而有某种经济,这种经济构成他们的世界的物质基础,就象资产阶级经济构成现今世界的物质基础一样。也许巴师夏的意思是说,建立在奴隶劳动上的生产方式是以某种掠夺制度为基础吧?如果是这样,他就处于危险的境地了。既然象亚里士多德那样的思想巨人在评价奴隶劳动时都难免发生错误,那末,象巴师夏这样的经济学侏儒在评价雇佣劳动时怎么会正确无误呢?——借这个机会,我要简短地回答一下美国一家德文报纸在我的《政治经济学批判》一书出版时(1859年)对我的指责。在那本书中我曾经说过,一定的生产方式以及与它相适应的生产关系,简言之,“社会的经济结构,是有法律的和政治的上层建筑竖立其上并有一定的社会意识形式与之相适应的现实基础”,“物质生活的生产方式制约着整个社会生活、政治生活和精神生活的过程”[45]。可是据上述报纸说,这一切提法固然适用于物质利益占统治地位的现今世界,但却不适用于天主教占统治地位的中世纪,也不适用于政治占统治地位的雅典和罗马。首先,居然有人以为这些关于中世纪和古代世界的人所共知的老生常谈还会有人不知道,这真是令人惊奇。很明白,中世纪不能靠天主教生活,古代世界不能靠政治生活。相反,这两个时代谋生的方式和方法表明,为什么在古代世界政治起着主要作用,而在中世纪天主教起着主要作用。此外,例如只要对罗马共和国的历史稍微有点了解,就会知道,地产的历史构成罗马共和国的秘史。而从另一方面说,唐·吉诃德误认为游侠生活可以同任何社会经济形式并存,结果遭到了惩罚。}
    
    商品世界具有的拜物教性质或劳动的社会规定所具有的物的外观,怎样使一部分经济学家受到迷惑,也可以从关于自然在交换价值的形成中的作用所进行的枯燥无味的争论中得到证明。既然交换价值是表示消耗在物上的劳动的一定社会方式,它就象汇率一样并不包含自然物质。
    
    由于商品形式是资产阶级生产的最一般的和最不发达的形式(所以它早就出现了,虽然不象今天这样是统治的、从而是典型的形式),因而,它的拜物教性质显得还比较容易看穿。但是在比较具体的形式中,连这种简单性的外观也消失了。货币主义的幻觉是从哪里来的呢?是由于货币主义没有看出:金银作为货币代表一种社会生产关系,不过采取了一种具有奇特的社会属性的自然物的形式。而蔑视货币主义的现代经济学,一当它考察资本,它的拜物教不是也很明显吗?认为地租是由土地而不是由社会产生的重农主义幻觉,又破灭了多久呢?
    
    为了不致涉及以后的问题,这里仅仅再举一个关于商品形式本身的例子。假如商品能说话,它们会说:我们的使用价值也许使人们感到兴趣。作为物,我们没有使用价值。作为物,我们具有的是我们的价值。我们自己作为商品物进行的交易就证明了这一点。我们彼此只是作为交换价值发生关系。现在,让我们听听经济学家是怎样说出商品内心的话的:
    
    \texttt{“价值〈交换价值〉是物的属性,财富〈使用价值〉是人的属性。从这个意义上说,价值必然包含交换,财富则不然。”\footnote{《评政治经济学上的若干用语的争论,特别是有关价值、供求的争论》1821年伦敦版第16页。}“财富〈使用价值〉是人的属性,价值是商品的属性。人或共同体是富的;珍珠或金刚石是有价值的……珍珠或金刚石作为珍珠或金刚石是有价值的。”\footnote{赛·贝利《对价值的本质、尺度和原因的批判研究》第165页及以下各页。}}
    
    直到现在,还没有一个化学家在珍珠或金刚石中发现交换价值。可是那些自命有深刻的批判力、发现了这种化学物质的经济学家,却发现物的使用价值同它们的物质属性无关,而它们的价值倒是它们作为物所具有的。在这里为他们作证的是这样一种奇怪的情况:物的使用价值对于人来说没有交换就能实现,就是说,在物和人的直接关系中就能实现;相反,物的价值则只能在交换中实现,就是说,只能在一种社会的过程中实现。在这里,我们不禁想起善良的道勃雷,他教导巡丁西可尔说[46]:
    
    \texttt{“一个人长得漂亮是环境造成的,会写字念书才是天生的本领”。\footnote{《评政治经济学上的若干用语的争论》一书的作者和赛·贝利责备李嘉图,说他把交换价值从一种只是相对的东西变成一种绝对的东西。恰恰相反,李嘉图是把金刚石、珍珠这种物在作为交换价值时所具有的表面的相对性,还原为这种外表所掩盖的真实关系,还原为它们作为人类劳动的单纯表现的相对性。如果说李嘉图派对贝利的答复既粗浅而又缺乏说服力,那只是因为他们在李嘉图本人那里找不到关于价值和价值形式即交换价值之间的内部联系的任何说明。}}
        

\end{document}